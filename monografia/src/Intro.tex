\chapter{Introdu��o}
Cada vez mais a sociedade moderna est� dependente dos computadores para realiza��o de suas atividades do cotidiano. Com isso a internet se tornou um item essencial em qualquer lugar.

S�o compras realizadas e contas pagas sem a necessidade de sair de casa. Tudo que � necess�rio � um computador e uma conex�o com a internet. Esses servi�os que facilitam tanto a vida v�o acumulando informa��es sobre os usu�rios, como localiza��o, dados financeiros, e outros dados pessoais ou informa��es confidenciais e os usu�rios acreditam que essas informa��es est�o bem protegidas. Mas isso nem sempre � verdade.

Tome como exemplo a \emph{PSN} (\estrangeiro{PlayStation Netowork}), a rede utilizada pelo PlayStation 3 para organizar partidas \estrangeiro{online}, para comprar e baixar jogos, comunica��o entre os usu�rios, dentre outros recursos. Recentemente uma vulnerabilidade foi encontrada nessa rede e v�rios dados de usu�rios foram obtidos por quem a explorou. Entre esses dados estavam informa��es sobre os cart�es de cr�dito utilizados para fazer compras na PSN.

O caso da PSN mostra que mesmo os maiores servi�os existentes est�o sujeitos a todo tipo de vulnerabilidade e nem sempre ferramentas tradicionais como \estrangeiro{firewalls} e criptografia s�o o suficiente para protege-las.

Os \estrangeiro{scanners} de vulnerabilidade s�o ferramentas que fazem a an�lise das aplica��es \estrangeiro{web} de forma autom�tica. S�o capazes de detectar e reportar vulnerabilidades de seguran�a\cite{Tania:2010:Scanners}.

O objetivo deste trabalho � analisar a efici�ncia de alguns desses \estrangeiro{scanners} de vulnerabilidade para encontrar um tipo de vulnerabilidade espec�fico: \estrangeiro{XPath Injection}.