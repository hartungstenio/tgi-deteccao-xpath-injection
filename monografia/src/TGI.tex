% Trabalho de Gradua��o Interdisciplinar - Christian Hartung
\documentclass[times]{abnt}

\usepackage[brazil]{babel}
\usepackage[latin1]{inputenc}
\usepackage[T1]{fontenc}
\usepackage{color}
\usepackage{graphics}
\usepackage[pdftex]{graphicx}
\usepackage{float}
\usepackage{booktabs}
\usepackage{mdwlist}
\usepackage{booktabs}
\usepackage[num]{abntcite}
\usepackage{url}

% A grossura padr�o da linha de assinatura � 0. Aumento para 0.4, s� para ela aparecer
\setlength{\ABNTsignthickness}{0.4pt}
\setlength{\ABNTsignskip}{2cm}
\setlength{\ABNTsignwidth}{6.5cm}

%%%%%%%%%%%%%%%%%%%% Comandos utilizados

% Texto em outro idioma
\newcommand{\estrangeiro}[1]{\textit{#1}}

% Insere uma figura
\newcommand{\figura}[3]{
 \begin{figure}[h]
  \centering
  \includegraphics[width=#3]{img/#1}%
  \caption{#2}%
  \label{fig:#1}%
 \end{figure}
}


\autor{Christian Hartung}
\titulo{Uma avalia��o de Scanners de Vulnerabilidade na detec��o de XPath Injection em Aplica��es Web}
\comentario{Trabalho de Gradua��o Interdisciplinar apresentado na Faculdade de Tecnologia - FT como requisito de conclus�o do curso de Tecnologia em An�lise e Desenvolvimento de Sistemas.}
\instituicao{Universidade Estadual de Campinas - UNICAMP\par
 Faculdade de Tecnologia - FT}
\orientador[Orientadora:]{\small Prof. Tania Basso}
\coorientador[Co-orientadora:]{\small Prof. Dr. Regina L. O. Moraes}
\local{Limeira}
\data{2011}

\begin{document}
\capa
\folhaderosto

% Errata

% Dedicat�ria

% Agradecimentos

% Ep�grafe

\sumario

\listoffigures

\listoftables

\begin{resumo}
 Aplica��es web podem apresentar brechas de seguran�a. Scanners de vulnerabilidades s�o ferramentas capazes de identificar essas brechas. O objetivo deste trabalho � avaliar a seguran�a de aplica��es web em rela��o � vulnerabilidade do tipo \emph{XPath Injection} e tamb�m avaliar a efic�cia de scanners de vulnerabilidades na detec��o dessas vulnerabilidades.
\end{resumo}

\begin{abstract}
 Web application can have security holes. Web Application vulnerability scanners are tools capable of identifying those holes. This work tries to evaluate the security of web application with respect of XPath Injection vulnerability and also assess the effectiveness of vulnerability scanners to detect these vulnerabilities. 
\end{abstract}

\chapter{Aplica��es e Servi�os Web} \label{chap:WebApps}
Nos �ltimos anos, muito se tem falado sobre as vantagens de se utilizar a chamada \emph{cloud computing} (computa��o em nuvem) como arquitetura de TI (Tecnologia da Informa��o) \cite{Cloud:computerworld} \cite{Cloud:idgnow} \cite{Cloud:info:ml} \cite{Cloud:info:entrevista} \cite{Cloud:olhardigital}. Essa arquitetura permite que as aplica��es sejam armazenadas em centros de armazenamento de terceiros\cite{Silva-Fabio-Rodrigues:2010:CloudComputing} e utilizadas por meio de m�quinas clientes, reduzindo, assim, os custos com aquisi��o e manuten��o de servidores, dispositivos para armazenamento, \estrangeiro{backups}, etc. O �nico requisito � que o cliente tenha um navegador de internet instalado.

A vantagem da utiliza��o da computa��o em nuvem est� no fato de o usu�rio, ao inv�s de instalar, por exemplo, um editor de textos, uma planilha eletr�nica ou um editor apresenta��es em seu computador, poder utilizar esses softwares como um servi�o armazenado em um provedor. Todos os recursos de hardware necess�rios para executar esses softwares, o espa�o para armazenamento de documentos, e outros recursos s�o responsabilidade desse provedor. O usu�rio precisa apenas utilizar o servi�o por meio do navegador de internet. Aplica��es com essa caracter�stica de serem acessadas pelo navegador s�o conhecidas como \emph{aplica��es web}.

Mas nem sempre uma aplica��o \estrangeiro{web} � poss�vel ou vantajoso. Se o servi�o em quest�o precisa ser acessado de um \estrangeiro{smartphone}, onde a experiencia com o navegador � mais limitada, pelo tamanho da tela ser menor, o poder de processamento reduzido, etc., ou o usu�rio n�o possui uma conex�o r�pida o bastante para acessar o servi�o, ou ainda o usu�rio prefere ter um aplicativo instalado para utilizar o servi�o, utilizar a aplica��o \estrangeiro{web} diretamente se tornaria um problema.

Nesses casos, o provedor do servi�o pode disponibilizar uma API (\estrangeiro{Application Programming Interface}, ou Interface de Programa��o de Aplica��es), para que seja desenvolvido um aplicativo que utilize o servi�o em quest�o. Essa API � chamada \emph{web service} (servi�o \estrangeiro{web}).

As se��es seguintes apresentam uma breve introdu��o sobre essas duas tecnologias (aplica��es \estrangeiro{web} e \estrangeiro{web services}): como funcionam, suas vantagens e desvantagens.

%%%%%%%%%%%%%%%%%%%%%%%%%%%%%%%%%%%%%%%%%%%%%%%%%%%%%%%%%%
\section{Aplica��es \estrangeiro{web}}
Uma aplica��o \estrangeiro{web} � qualquer aplica��o que utilize um navegador de internet como cliente\cite{About:WebApps}. Apesar de o termo \estrangeiro{cloud computing} ter surgido em meados de 2006\cite{zdnet:IntroCloud}, as primeiras aplica��es \estrangeiro{web} surgiram por volta de 1987, antes mesmo de a internet se popularizar\cite{About:InternetHistory} 

Aplica��es \estrangeiro{web} funcionam utilizando uma arquitetura cliente-servidor. Nessa arquitetura, o cliente (navegador de internet) solicita um recurso do servidor. O servidor, por sua vez, responde com uma representa��o do recurso solicitado em um formato que o navegador pode entender: a HTML (\estrangeiro{Hypertext Markup Language}),  Linguagem de Marca��o de Hipertexto. 

A HTML sozinha � uma linguagem limitada\cite{About:InternetHistory}. Tudo que ela faz � dizer ao navegador como interpretar o conte�do da p�gina e ent�o, apenas p�ginas est�ticas podem ser constru�das com ela. Devido a essa limita��o da HTML, as primeiras aplica��es \estrangeiro{web} n�o passavam de p�ginas exibidas em determinada sequ�ncia, baseadas em dados de entrada do usu�rio. Com o tempo, novas tecnologias surgiram para permitir que mais pudesse ser feito com a HTML. A principal delas surgiu em 1995: as linguagens \estrangeiro{client-side}.

Linguagens \estrangeiro{client-side} s�o linguagens de programa��o que s�o executadas no lado cliente da aplica��o, ou seja, no navegador de internet. Elas permitiram que alguma l�gica fosse adicionada �s p�ginas HTML, tornando as aplica��es Web mais interativas. Atualmente a principal linguagem \estrangeiro{client-side} � o JavaScript, j� que � a linguagem de script padr�o do HTML5\footnote{O HTML5 � a vers�o do HTML em desenvolvimento}\cite{whatwg:HTML}.

Com o uso de JavaScript, as aplica��es \estrangeiro{web} t�m se tornado bastante din�micas e j� se assemelham �s aplica��es tradicionais (aplicativos que s�o executados no \estrangeiro{desktop}), algumas at� substituindo essas aplica��es tradicionais. Atualmente, milh�es de pessoas utilizam aplica��es \estrangeiro{web} quando acessam o \estrangeiro{webmail}, o bate-papo, compras online, \estrangeiro{internet banking}, etc

\subsection{Arquitetura de uma Aplica��o \estrangeiro{web}}
Uma aplica��o \estrangeiro{web} utiliza uma arquitetura multicamadas. As principais camadas s�o a camada de apresenta��o, camada de neg�cios e camada de persist�ncia.

\subsubsection{Camada de Apresenta��o}
A \emph{camada de apresenta��o} � a interface da aplica��o apresentada aos usu�rios. � a parte da aplica��o exibida no navegador \estrangeiro{web}, com a qual o usu�rio pode interagir.

\subsubsection{Camada de Neg�cios}
� a camada respons�vel pela l�gica de neg�cios da aplica��o. � a camada que serve como intermedi�ria entre a as camadas de apresenta��o e de persist�ncia, processando os dados que o usu�rio entra e passando-os para a camada de persist�ncia, ou lendo os dados que o usu�rio solicitou e exibindo-os na camada de apresenta��o.

Essa camada normalmente � formada por um servidor HTTP (\estrangeiro{Hypertext Transfer Protocol}, - Protocolo de Transfer�ncia de Hipertexto), ou servidor \estrangeiro{web}, como � mais conhecido, que recebe as requisi��es dos clientes e respondem enviando representa��es do que o cliente pediu. Essas representa��es podem ser imagens, v�deos, documentos HTML, etc.

O HTTP � o protocolo padr�o para transfer�ncia de dados pela internet e est� atualmente na vers�o 1.1\cite{RFC:2616}. A figura \ref{fig:TemplateHTTP} mostra o formato de uma requisi��o HTTP.

\figura{TemplateHTTP}{Formato de uma requisi��o HTTP}{7cm}

Na Figura \ref{fig:TemplateHTTP}, ``\estrangeiro{Host}'', ``\estrangeiro{Cookie}'' e ``\estrangeiro{From}'' s�o os cabe�alhos da requisi��o.  Uma requisi��o pode ter qualquer quantidade de cabe�alhos, padronizados ou personalizados.  Eles s�o utilizados para enviar meta-informa��es com cada requisi��o, ou seja, enviar informa��es adicionais sobre o recurso que est� sendo transmitido. Exemplos de cabe�alhos comumente usados s�o:
\begin{description*}
	\item[Host:] identifica o dom�nio que est� sendo acessado;
	\item[From:] cont�m o \estrangeiro{e-mail} do usu�rio que fez a requisi��o;
	\item[Cookie:] cont�m informa��es de autentica��o na aplica��o em quest�o;
	\item[Content-Type:] informa o tipo de dados em que o conte�do da requisi��o est� codificado;
	\item[Content-Length:] informa o tamanho do conte�do da requisi��o;
\end{description*}

O \emph{m�todo} utilizado na requisi��o indica o que o servidor deve fazer com o recurso transmitido. Dois m�todos principais s�o utilizados em uma aplica��o web: \emph{GET} e \emph{POST}.

O cliente utiliza o m�todo \texttt{GET} quando deseja obter recursos do servidor. A figura \ref{fig:HTTP-GET} mostra um exemplo de requisi��o \texttt{GET} para obter o recurso \texttt{usuarios/chrishartung} do servidor respons�vel pelo dom�nio \texttt{exemplo.com}. A p�gina acessada para isso � \url{http://exemplo.com/usuarios/chrishartung}.

\figura{HTTP-GET}{Requisi��o HTTP GET}{5cm}

J� o m�todo \texttt{POST} � utilizado quando o cliente deseja enviar informa��es para o servidor. Um uso comum dessa requisi��o na internet � para o preenchimento de formul�rios. A figura \ref{fig:HTTP-POST} mostra um exemplo de requisi��o \texttt{POST} que altera a institui��o do usu�rio \texttt{chrishartung}. Observe que neste caso foi informado o tipo dos dados em que o conte�do foi codificado (\emph{application/x-www-form-urlencoded}) e o tamanho da mensagem. 

\figura{HTTP-POST}{Requisi��o HTTP POST}{11cm}

O tipo \texttt{application/x-www-form-urlencoded} � utilizado como padr�o no HTML\cite{W3C:HTML4} para envio de formul�rios. Neste formato, apenas caracteres ASCII\footnote{O \estrangeiro{American Standard Code for Information Interchange} � uma codifica��o de caracteres baseado no alfabeto ingl�s. Consiste de 128 caracteres, sendo 94 vis�veis, 33 caracteres de controle (caracteres que informam como o texto deve ser processado) e o caractere de espa�o.} podem ser utilizados, ent�o caracteres n�o-ASCII devem ser codificados no formato $\%HH$, onde \texttt{HH} s�o dois d�gitos hexadecimais, e espa�os s�o substitu�dos por $+$. Os campos s�o enviados como pares \texttt{nome=valor}, separados pelo caractere \texttt{\&}.

O servidor \estrangeiro{web} sozinho s� � capaz de enviar conte�do est�tico para o cliente. O lado servidor da aplica��o (\estrangeiro{server-side}) tamb�m entende alguma linguagem de programa��o. Essa linguagem de programa��o � respons�vel por construir a resposta para a requisi��o dinamicamente, ler e  gravar informa��es no servidor de dados, etc.

Exemplos comuns de linguagens rodando no servidor s�o o PHP (\estrangeiro{Hypertext Preprocessor}), ASP (\estrangeiro{Active Server Pages}), JSP (\estrangeiro{Java Server Pages}), ColdFusion, etc.

\subsubsection{Camada de Persist�ncia}
A maioria das aplica��es \estrangeiro{web} lida com armazenamento de informa��es, tempor�ria ou permanentemente. Isso pode ser feito utilizando Sistemas Gerenciadores de Banco de Dados (SGBD's) ou outros sistemas mais simples, que podem chegar a arquivos de texto. A camada de persist�ncia � respons�vel por armazenar essas informa��es.

Servidores de dados que comp�em a camada de persist�ncia s�o os computadores mais protegidos da empresa, j� que armazenam os dados de todos os usu�rios da aplica��o. A configura��o desses servidores tem um impacto muito grande no desempenho da aplica��o. Essas s�o as m�quinas que possuem hardware mais moderno e recebem investimentos com maior frequ�ncia.

\subsection{Vantagens das aplica��es Web}
Aplica��es \estrangeiro{web} vem ganhando muita aten��o tanto em ambientes corporativos como no uso pessoal. Isso porque oferecem algumas vantagens sobre as aplica��es de \textit{desktop}\cite{AJAX-PHP-WebApps}:
\begin{description*}
	\item[F�cil distribui��o:] Basta um navegador e uma conex�o com a internet ou intranet. N�o � necess�rio que o departamento de TI instale a aplica��o em cada esta��o da empresa;
	\item[Independente de plataforma:] O usu�rio final pode usar Windows, Mac, Linux, Solaris, iPhone, etc. como sistema operacional e qualquer navegador dispon�vel\footnote{Se a aplica��o foi constru�da utilizando os padr�es da W3C corretamente}.
	\item[Dados centralizados:] N�o h� dados salvos na m�quina do usu�rio. Tudo est� no servidor.
\end{description*}

Alguns problemas tamb�m surgem ao utilizar essa arquitetura. Os principais deles s�o:
\begin{description*}
	\item[Depend�ncia da rede:] Todos os usu�rios do sistema dependem da rede ou internet para utiliz�-lo. Se ocorrer algum problema com o link da internet utilizado, todos os usu�rios ficar�o comprometidos.
	\item[Depend�ncia do servidor:] Se o servidor ficar indispon�vel, o sistema todo tamb�m fica.
\end{description*}

%%%%%%%%%%%%%%%%%%%%%%%%%%%%%%%%%%%%%%%%%%%%%%%%%%%%%%%%%%%%%%%%%%%%%%%
\section{\estrangeiro{Web services}}
Um \emph{web service} (servi�o web) � um software que implementa diversas tecnologias para transfer�ncia e representa��o de dados que, quando combinadas, permitem a comunica��o entre os diversos dispositivos conectados em uma rede\cite{W3C:ws-arch}.

Diferentemente das aplica��es \estrangeiro{web}, onde �nico cliente � o navegador de internet, um \estrangeiro{web service} pode ter os mais variados tipos de clientes, como aplicativos pr�prios, aplicativos de terceiros, linha de comando, celulares e at� mesmo aplica��es \estrangeiro{web} ou outros \estrangeiro{web services}.

Um \estrangeiro{web service} � um software que fica no servidor e que responde a requisi��es dos clientes.

\figura{WebServiceLoja}{Exemplo de aplica��o web utilizando web services}{10cm}

A figura \ref{fig:WebServiceLoja} mostra como funciona uma aplica��o \estrangeiro{web} de cota��o de pre�os que consome os servi�os oferecidos pelos diversos \estrangeiro{web services} das lojas virtuais. A \emph{Aplica��o Cliente} � o navegador de internet rodando no computador do usu�rio. Quando o usu�rio consulta o pre�o de um produto, a aplica��o \estrangeiro{web} \emph{Sistema de Pesquisa} consulta nos \estrangeiro{web services} das lojas virtuais a cota��o deste produto. Com as respostas obtidas dos \estrangeiro{web services}, o Sistema de Pesquisa monta a p�gina com as informa��es que o cliente deseja.

Antes de come�ar a implementa��o de um \estrangeiro{web service}, � preciso definir as tecnologias que ser�o utilizadas para representar os dados e para transferi-los.

\subsection{Representa��o dos dados}
A tecnologia para representa��o dos dados � o idioma que o web service fala. De nada adianta um professor falar latim se seus alunos entendem apenas portugu�s. Da mesma forma, de nada adianta o servidor enviar os dados utilizando uma representa��o que os clientes n�o s�o capazes de interpretar.

N�o existe nenhuma norma que obrigue a utilizar determinada representa��o para desenvolvimento de um \estrangeiro{web service}\footnote{A natureza do SOAP induz � utiliza��o de XML, mas n�o � obrigat�rio usar SOAP para cria��o de um \estrangeiro{web service}.}. Atualmente os formatos mais utilizados s�o o XML (\estrangeiro{Extensible Markup Language}, ou Linguagem de Marca��o Extens�vel) e o JSON (\estrangeiro{JavaScript Object Notation}, ou Nota��o de Objeto JavaScript).

\subsubsection{XML}
A XML � uma linguagem de marca��o padronizada pela W3C\cite{W3C:XML}, que a define como um formato para representar informa��es estruturadas na forma de texto. Sua ado��o se d� principalmente pelo fato de ter sido criada pensando em internet, em facilidade de uso, em ser independente de plataforma ou linguagem de programa��o e em ser leg�vel para m�quinas e humanos.

\figura{XMLUsuarios}{Exemplo de documento XML}{7cm}

A figura \ref{fig:XMLUsuarios} mostra um exemplo de documento XML que representa um usu�rio do sistema. A primeira linha cont�m a declara��o XML, que identifica a vers�o da especifica��o da XML sendo utilizada (1.0) e o conjunto de caracteres utilizado (UTF-8). Na segunda linha est� o elemento raiz do documento (``usuarios''). Os documentos XML tamb�m podem ser representados em estrutura de �rvore\cite{W3Schools:XML}, que tem esse elemento como raiz. As linhas seguintes definem elementos filhos da raiz, ou at� mesmo filhos dos filhos da raiz. A �ltima linha encerra o elemento raiz, encerrando o documento. A figura \ref{fig:XMLUsuariosArvore} mostra esse mesmo documento em formato de �rvore.

\figura{XMLUsuariosArvore}{Representa��o de documento XML em �rvore}{7cm}

Os documentos XML s�o formados por duas constru��es principais: as \estrangeiro{tags} (marcadores) e seus atributos. Uma \estrangeiro{tag} � um elemento entre os sinais de ``<'' e ``>'' e os atributos s�o pares nome/valor que ficam dentro da \estrangeiro{tag}. No exemplo da Figura \ref{fig:XMLUsuarios}, ``usuarios'', ``usuario'', ``nome'' e ``nascimento'' s�o \estrangeiro{tags}, enquanto ``login'' � um atributo da \estrangeiro{tag} ``usuario''. A especifica��o do XML n�o define nenhuma \estrangeiro{tag} ou atributo, cabendo ao desenvolvedor do sistema defin�-los.

\subsubsection{JSON}
Apesar de o XML ser o formato mais utilizado para transfer�ncia de dados\cite{Costa-Guilherme:2009:SOA}, ele � um formato que possui muitas regras, o que o torna pesado.

Pensando nisso o formato JSON foi criado e padronizado\cite{IETF:JSON}, com base na sintaxe para cria��o de objetos da linguagem JavaScript. A figura \ref{fig:JSONUsuarios} mostra como ficaria o documento XML da figura \ref{fig:XMLUsuarios}, que representa um usu�rio do sistema, utilizando o formato JSON.

\figura{JSONUsuarios}{Exemplo de objeto JSON}{5cm}

O JSON � formado por um conjunto de pares nome/valor. O nome � uma \estrangeiro{string} (definido como uma sequ�ncia de caracteres entre aspas) e o valor pode ser qualquer um dos tipos de dados suportados:
\begin{description*}
	\item[string:] cadeias de caracteres entre aspas;
	\item[n�meros:] valores num�ricos, com ou sem casas decimais;
	\item[booleanos:] \estrangeiro{true} (verdadeiro) ou \estrangeiro{false} (falso);
	\item[\estrangeiro{arrays}:] conjunto de valores delimitado por colchetes, como \texttt{[1, [3,4], \{``nome'':``fulano''\}]};
	\item[objetos:] conjunto de pares nome/valor;
	\item[vazio:] nenhum valor, representado pela palavra \texttt{null};
\end{description*}

Estudos indicam que o formato JSON pode ser analisado at� cem vezes mais r�pido que o XML\cite{JSONxXML}.

\subsection{Transfer�ncia das informa��es}
Al�m de falarem o mesmo idioma, a forma como a informa��o � transmitida tamb�m deve ser compat�vel entre o cliente e o servidor.

De nada adianta o cliente fazer requisi��es utilizando um protocolo que o servidor n�o entende, da mesma forma que de nada adianta o servidor responder �s requisi��es de uma forma que o cliente n�o possa interpretar.

Atualmente existem duas formas para transfer�ncia de dados que tem sido difundidas: o \emph{SOAP} (\estrangeiro{Simple Object Access Protocol}, ou Protocolo Simples de Acesso a Objetos) e o \emph{REST} (\estrangeiro{Representational State Transfer}, ou Transfer�ncia de Estado Representacional).

\subsubsection{SOAP}
O SOAP � um protocolo para troca de informa��es por meio de mensagens XML sobre um protocolo de aplica��o, normalmente o HTTP.  � um dos protocolos mais utilizados para cria��o de \estrangeiro{web services}, principalmente pelo fato de ser baseado em XML, o que o torna compat�vel com qualquer linguagem de programa��o, al�m de ser uma recomenda��o da W3C\cite{W3C:SOAP}.

Uma comunica��o utilizando SOAP funciona da seguinte forma: quando o cliente se conecta ao servi�o, obt�m um documento listando os m�todos dispon�veis, seus par�metros e tipos de dados. Este documento � chamado \emph{WSDL} (\estrangeiro{Web Services Description Language}, ou Linguagem de Descri��o de Servi�os Web) , uma linguagem tambem baseada em XML. As figuras \ref{fig:WSDL-Request} e \ref{fig:WSDL-Response} s�o partes de um documento WSDL que define uma mensagem para obter os dados de uma cidade (m�todo \texttt{GetCidade}). Na figura \ref{fig:WSDL-Request} est� o trecho que define a \emph{requisi��o}, ou seja, o formato da mensagem que deve ser enviado para o servidor. A figura \ref{fig:WSDL-Response} mostra o trecho que define o formato da \emph{resposta} que o servidor envia.

\figura{WSDL-Request}{Documento WSDL para um servi�o que obt�m dados de uma cidade - Requisi��o}{7cm}

\figura{WSDL-Response}{Documento WSDL para um servi�o que obt�m dados de uma cidade - Resposta}{8cm}

Tendo este documento, o aplicativo cliente � capaz de fazer a requisi��o SOAP propriamente dita, enviando uma mensagem HTTP semelhante � da figura \ref{fig:SOAP}. Na linha 6 est� a chamada ao m�todo \texttt{GetCidade} e na linha 7 est� sendo passado o par�metro \texttt{Cidade} com um valor informado pelo usu�rio.

\figura{SOAP}{Requisi��o SOAP}{9cm}

\subsubsection{REST}
O REST n�o � um padr�o, mas um conceito. � um conjunto de \estrangeiro{constraints} (restri��es) que, quando seguidas, tornam o servi�o \estrangeiro{RESTful}\cite{REST}.

Diferente do SOAP, que cria um novo protocolo sobre o HTTP para a transfer�ncia da mensagem, em um servi�o \estrangeiro{RESTful} toda mensagem � transmitida diretamente no HTTP\footnote{O conceito foi desenvolvido sobre o protocolo HTTP, mas nada impede que seja utilizado outro protocolo, desde que as \estrangeiro{constraints} possam ser implantadas.}.

O REST trabalha com a transfer�ncia de representa��es de recursos de um lado para o outro da aplica��o, acessadas atrav�s de identificadores. Um recurso � um elemento conceitual da aplica��o. Pode ser um usu�rio, uma cidade, um documento, pode estar armazenado em mem�ria, em um banco de dados, em um documento XML. A representa��o � uma materializa��o desse conceito de uma forma que o aplicativo possa entender, como um documento XML ou JSON. A tabela \ref{tab:elementos-rest} mostra esses elementos e um exemplo deles em um servi�o para controle de usu�rios.

\begin{table}[h]
 \centering
 \begin{tabular}{lp{10cm}}
 \toprule
  \textbf{Elemento} & \textbf{Exemplo} \\
  \midrule
  Recurso & Usu�rio \\
  \midrule
  Identificador & URI (\estrangeiro{Uniform Resource Identifier}, ou Identificador Uniforme de Recursos) onde o recurso pode ser encontrado. \texttt{/usuarios/chrishartung} \\
  \midrule
  Representa��o & Documento JSON representando um usu�rio. \texttt{\{``usuario'':\{``login'':``chrishartung'', ``nome'':``Christian Hartung'', ``nascimento'':``1988-03-08''\}} \\
  \bottomrule
 \end{tabular}
 
 \caption{Elementos de uma aplica��o \estrangeiro{RESTful}}
 \label{tab:elementos-rest}
\end{table}

As \estrangeiro{constraints} que devem ser respeitadas s�o\cite{REST}:
\begin{description*}
	\item[Cliente-servidor:] Deve haver uma clara separa��o entre as responsabilidades do cliente e do servidor. O servidor n�o deve se preocupar com como o recurso ser� exibido, e o cliente n�o deve se preocupar em como o recurso ser� armazenado. Isso permite que as duas partes fiquem mais simples e evoluam de forma independente.
	\item[\estrangeiro{Steteless}:] O servidor n�o deve guardar nenhuma informa��o intermedi�ria sobre o processamento. Toda essa informa��o fica no cliente, e � transmitida na requisi��o sempre que necess�ria.
	\item[\estrangeiro{Cache}:] Deve haver uma marca��o que indica se a resposta pode ser guardada pelo cliente para uso futuro ou n�o.
	\item[Interface uniforme:] Ter uma interface uniforme permite que o sistema fique mais desacoplado e ajuda que cada componente evolua de forma independente. Neste ponto est�o as URI's.
	\item[Multicamadas:] Um cliente nunca pode assumir que est� conectado diretamente ao servidor. Podem existir m�quinas intermedi�rias para aumentar a seguran�a e escalabilidade do servi�o, como \estrangeiro{firewalls}, servidores \estrangeiro{proxy}, etc. Da mesma forma, o servidor n�o pode assumir que est� conectado diretamente ao cliente.
	\item[C�digo sob demanda:] Esta \estrangeiro{constraint} � opcional e diz que o cliente deve ser capaz de executar c�digo enviado pelo servidor.
\end{description*}

O exemplo mais utilizado de servi�o \estrangeiro{RESTful} � a \emph{rede mundial de computadores} (\estrangeiro{World Wide Web}): a internet\cite{REST}.

\subsection{Vantagens e Desvantagens}
Utilizando \estrangeiro{web services}, � poss�vel dividir o processamento que uma aplica��o precisa realizar entre os clientes e o servidor, tornando os aplicativos nas duas partes mais leves.

Por serem baseados em padr�es abertos, a grande maioria das linguagens de programa��o possui suas API's para cria��o de \estrangeiro{web services}, e estas s�o compat�veis entre si, portanto o cliente e o servidor podem ser escritos em linguagens diferentes.

Quando utilizando SOAP, uma limita��o � que os dados s� podem ser transmitidos em XML, dando menor flexibilidade para a aplica��o.

Outra preocupa��o que se deve ter � com como os dados trafegar�o. Como o uso de \estrangeiro{web services} envolve rede, ou mesmo a internet, � preciso que as requisi��es sejam criptografadas para evitar que os dados sejam obtidos por outros.
\chapter{XPath e XPath Injection} \label{chap:XPath}
A maioria das aplica��es web utilizam sistemas gerenciadores de bancos de dados (SGBD) para armazenar seus dados\cite{devWorks:XPath}, especialmente porque esses sistemas ajudam a garantir a integridade e facilitam a manipula��o das informa��es armazenadas. Por�m, utilizar um SGBD nem sempre � a melhor solu��o. Manter um SGBD n�o � uma tarefa barata. � preciso um especialista em cada SGBD utilizado, monitoramento constante, aquisi��o de servidores e at� mesmo licen�as de uso desses sistemas, etc. Para uma aplica��o pequena, por exemplo, um  esses gastos n�o compensariam. Tamb�m, se alguma customiza��o for feita para um caso espec�fico, n�o compensaria alterar a estrutura da base de dados da aplica��o para contemplar esse �nico caso. Para esses tipos de situa��es, documentos XML se tornam uma op��o interessante, com a vantagem de serem leg�veis para a aplica��o e para seus usu�rios e desenvolvedores.

\section{Entendendo o XPath}
A XML Path Language, ou XPath, como � mais conhecida, � uma linguagem de consulta capaz de localizar elementos de um documento XML e at� mesmo realizar c�lculos com esses elementos\cite{W3C:XPath}.

Em uma consulta XPath, o documento XML � visto como uma �rvore, onde cada tag, cada atributo, cada texto, cada coment�rio, etc. � tratado como um n�\cite{W3Schools:XPath}. A linguagem XPath permite navegar por essa �rvore, selecionando um conjunto de n�s baseado em uma s�rie de crit�rios. Desde 1999, XPath � uma recomenda��o da W3C\cite{W3C:XPath}. Suas principais express�es s�o:
\begin{description*}
  \item[\textit{nome}] - Seleciona todos os filhos com o nome dado
  \item[/] - Seleciona a partir do n� atual
  \item[//] - Seleciona todos os n�s do documento que satisfa�am o crit�rio, partindo do atual
  \item[.] - Seleciona o n� atual
  \item[..] - Seleciona o pai do n� atual
  \item[@] - Seleciona um atributo
\end{description*}

Considere o documento XML da figura \ref{fig:ExemploXPath}, extra�do de uma aplica��o real. Este documento � utilizado para controlar a aprova��o de cota��es de compra em um sistema ERP (\estrangeiro{Enterprise Resource Planning}, sistemas integrados de gest�o empresarial), baseando-se no valor da cota��o. Cada elemento VALOR lista os logins dos usu�rios que podem aprovar cota��es com valores iguais ou inferiores ao definido valor pelo atributo VALUE. No caso, o usu�rio EMILY pode aprovar cota��es de at� R\$500, o usu�rio TANIA pode aprovar cota��es de at� R\$999 e o usu�rio CHRISTIAN  pode aprovar cota��es de at� R\$1.000.000.

\figura{ExemploXPath}{Documento XML de exemplo para consultas}{7cm}

A seguir est�o algumas das principais express�es XPath aplicadas ao documento XML da figura \ref{fig:ExemploXPath}, junto de seus resultados:
\begin{description*}
  \item[VALCOMPRAS]\hfill \\
    Seleciona todos os filhos do elemento raiz (VALCOMPRAS)
  \item[/VALCOMPRAS]\hfill \\
    Seleciona o elemento raiz
  \item[VALCOMPRAS/VALOR]\hfill \\
    Seleciona todos os elementos VALOR filhos diretos de VALCOMPRAS
  \item[VALCOMPRAS//USER]\hfill \\
    Seleciona todos os elementos USER descendentes de VALCOMPRAS. N�o � necess�rio ser um filho direto
  \item[//@VALUE]\hfill \\
    Seleciona todos os atributos VALUE do documento
\end{description*}
Observe que as express�es s�o utilizadas para criar caminhos atrav�s do documento XML, da� o nome \textit{XML Path Language} (Linguagem de Caminhos pelo documento XML).

Para tornar o caminho at� determinado elemento mais espec�fico, � poss�vel filtrar as express�es. A seguir est�o alguns exemplos de filtros (express�es entre colchetes) aplicados ao documento da figura \ref{fig:ExemploXPath} e o resultado da consulta.
\begin{description*}
  \item[VALCOMPRAS/VALOR[1]]\hfill \\
    Seleciona o primeiro elemento VALOR filho de VALCOMPRAS
  \item[VALCOMPRAS/VALOR[last()]]\hfill \\
    Seleciona o �ltimo elemento VALOR filho de VALCOMPRAS
  \item[VALCOMPRAS/VALOR[position() <= 2]]\hfill \\
    Seleciona os dois primeiros elementos VALOR filhos de VALCOMPRAS
  \item[//VALOR[@VALUE]]\hfill \\
    Seleciona todos os elementos VALOR que possuam um atributo VALUE
  \item[//USERS[USER]]\hfill \\
    Seleciona todos os elementos USERS que tenham uma tag USER como filho
  \item[VALCOMPRAS/VALOR[@VALUE >= 1000]//USER]\hfill \\
    Seleciona todos os elementos USER que descendentes de um elemento VALOR com o atributo VALUE maior ou igual a 1000
  \item[//VALOR[@VALUE < 100 or @VALUE > 500]]\hfill \\
    Seleciona todos os elementos VALOR que tenham o elemento VALUE menor que 100 ou maior que 500
\end{description*}

\section{XPath Injection}
Segundo \cite{WASC:XPath}, XPath Injection � uma t�cnica para explorar aplica��es que montam consultas XPath a partir de dados fornecidos pelo usu�rio. � uma forma de vulnerabilidade de inje��o de c�digo.

O c�digo na figura \ref{fig:ExemploCSharp}, escrito em C\#, mostra uma aplica��o que consulta o documento XML da figura \ref{fig:ExemploXPath} para obter uma lista com todos os usu�rios que podem aprovar cota��es de compra de um determinado valor.

\figura{ExemploCSharp}{C�digo C\# executando uma consulta XPath}{13cm}

Na linha 3 � feita a consulta XPath baseando-se no valor de um campo de entrada. Supondo que a cota��o de compras seja de R\$1000, essa consulta ficaria ``\texttt{//VALOR[@VALUE >= 1000]/USERS/USER/@NAME}'' o que retorna a lista de nomes corretamente.

Como n�o existe nenhuma valida��o deste campo antes de seu uso, o usu�rio pode informar qualquer conte�do para este campo. Por exemplo, se o par�metro tiver o formato 1000 or 1 = 1, a consulta ficaria ``\texttt{//VALOR[@VALUE >= 1000 or 1 = 1]/USERS/USER/@NAME}'' retornando os nomes de todos os usu�rios no documento, independente do valor permitido, j� que a condi��o 1 = 1 ser� sempre verdadeira.

Estudos da WASC\cite{WASC:Statistics} encontraram 64 vulnerabilidades do tipo \estrangeiro{XPath Injection} em 19 aplica��es. Nenhuma delas foi detectada pelos \estrangeiro{scanners} de vulnerabilidade utilizados por eles.

Essa vulnerabilidade possui nota 10 utilizando o sistema CVSS (\estrangeiro{Common Vulnerability Scoring System}, sistema que compara as efeitos causados pelas vulnerabilidades) e � classificada como urgente pelo \estrangeiro{Payment Card Industry Security Standards Council} (grupo que define padr�es de seguran�a para empresas que lidam com dados de cart�es de cr�dito e d�bito)\cite{PCI-DSS}

\section{Evitando XPath Injection}
A melhor maneira de evitar este tipo de ataque � assumir que todos os dados de entrada s�o suspeitos e precisam ser validados antes de serem utilizados na consulta XPath. Algumas poss�veis valida��es:
\begin{enumerate}
	\item Se os dados forem num�ricos, verificar se apenas n�meros e separadores de milhar e decimal est�o presentes;
	\item Verificar a presen�a de aspas no texto e remov�-las;
	\item Remover qualquer operador do XPath que exista nos dados de entrada;
\end{enumerate}
\section{Estudo}

\begin{frame}
 \frametitle{Objetivo}
 
 \begin{itemize}[<+->]
 	\item Avaliar um conjunto de \estrangeiro{scanners} de vulnerabilidade, verificando a efic�cia na detec��o de vulnerabilidades do tipo \estrangeiro{XPath Injection}
 \end{itemize}
\end{frame}

\chapter{\estrangeiro{Scanners} de vulnerabilidades} \label{chap:scanners}
\estrangeiro{Scanners} de vulnerabilidade s�o uma forma de analisar uma aplica��o de forma autom�tica em busca de vulnerabilidades\cite{nuno}. Eles s�o utilizados como uma forma de detectar problemas gerados pela m� codifica��o da aplica��o. A maioria dos \estrangeiro{scanners} de vulnerabilidade s�o ferramentas comerciais, mas existem alguns gratuitos e at� mesmo com o c�digo fonte dispon�vel.

Segundo \cite{Tania:2010:Scanners}, essas ferramentas funcionam em tr�s fases:
\begin{description*}
	\item[Configura��o:] Nesta fase o \estrangeiro{scanner} deve ser informado sobre a aplica��o que ele varrer�, as vulnerabilidades que ser�o buscadas nessa varredura e informa��es de conex�o com o servi�o (proxy - servidor intermedi�rio entre o cliente e a aplica��o). � a �nica fase em que � necess�ria interven��o do usu�rio.
	\item[Rastreamento:] O \estrangeiro{scanner} analisa o conte�do de cada p�gina da aplica��o em busca de \estrangeiro{links} e monta um mapa de como a aplica��o est� organizada.
	\item[Explora��o:] O \estrangeiro{scanner} faz a busca pelas vulnerabilidades, fazendo simula��es da intera��o com o usu�rio (simulando um clique em um \estrangeiro{link}, o preenchimento de um formul�rio, etc.). Ao final deste processo � gerado o relat�rio com todas as vulnerabilidades detectadas.
\end{description*}

Os algor�timos utilizados para detec��o das vulnerabilidades � pr�prio de cada \estrangeiro{scanner}, o que faz com que os resultados obtidos sejam variem de \estrangeiro{scanner} para \estrangeiro{scanner}. Isso nos leva a questionar qual dos resultados � o mais correto.
\subsection{Testes realizados}

\begin{frame}
 \frametitle{Utiliza��o dos \estrangeiro{scanners}}
 
 \begin{itemize}[<+->]
 	\item Configura��o dos \estrangeiro{scanners}
 	\item An�lise da aplica��o
 	\item An�lise do relat�rio
 	\item Divertido
 \end{itemize}
\end{frame}

\begin{frame}
 \frametitle{Valida��o dos resultados}
 
 \begin{itemize}[<+->]
 	\item Verifica��o de cada resultado
 	\item Teste de Caixa-branca
 	\item Teste de Caixa-preta
 \end{itemize}
\end{frame}
\subsection{Resultados}

\begin{frame}
 \frametitle{Utiliza��o dos \estrangeiro{scanners}}
 
 \begin{table}[h]
	 \centering
	 \begin{tabular}{lrrr}
	 \toprule
	  \textbf{\estrangeiro{Scanner}} & \textbf{Encontradas} & \textbf{Falta de Cobertura} \\
	  \midrule
	  Arachni & 3 & 29 \\
	  \midrule
	  w3af & 3 & 29 \\
	  \midrule
	  WebCruiser & 5 & 27 \\
	  \bottomrule
	 \end{tabular}
	\end{table}
\end{frame}
\chapter{Conclus�o}
Este estudo apresentou os resultados de tr�s scanners de vulnerabilidade na detec��o de uma amostra com mais de trinta vulnerabilidades comprovadas do tipo XPath Injection.

Apesar de os scanners terem conseguido n�meros melhores em outros tipos de vulnerabilidades, pode-ser perceber que n�o � t�o simples detectar XPath Injection em uma aplica��o.

Os resultados mostram que escolher um scanner de vulnerabilidade n�o � f�cil. Diversos fatores devem ser levados em considera��o, como o tempo necess�rio para scannear a aplica��o, n�mero de resultados encontrados, vulnerabilidades suportadas, etc.

Estudos futuros devem utilizar outros tipos de vulnerabilidades al�m do XPath Injection, e procurar outros scanners.

\bibliography{bibliografia}
\end{document}