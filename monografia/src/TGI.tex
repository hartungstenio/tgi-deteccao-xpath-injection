% Trabalho de Gradua��o Interdisciplinar - Christian Hartung
\documentclass[times]{abnt}

\usepackage[brazil]{babel}
\usepackage[latin1]{inputenc}
\usepackage[T1]{fontenc}
\usepackage{color}
\usepackage{graphics}
\usepackage[pdftex]{graphicx}
\usepackage{float}
\usepackage{booktabs}
\usepackage{mdwlist}
\usepackage{listings}
\usepackage{booktabs}
\usepackage[num]{abntcite}
\usepackage{url}

% A grossura padr�o da linha de assinatura � 0. Aumento para 0.4, s� para ela aparecer
\setlength{\ABNTsignthickness}{0.4pt}
\setlength{\ABNTsignskip}{2cm}
\setlength{\ABNTsignwidth}{6.5cm}

% Configura��es da package listings
\lstset{ %
basicstyle=\footnotesize,       % the size of the fonts that are used for the code
numbers=left,                   % where to put the line-numbers
numberstyle=\footnotesize,      % the size of the fonts that are used for the line-numbers
stepnumber=1,                   % the step between two line-numbers. If it's 1, each line 
                                % will be numbered
numbersep=6pt,                  % how far the line-numbers are from the code
backgroundcolor=\color{white},  % choose the background color. You must add \usepackage{color}
showspaces=false,               % show spaces adding particular underscores
showstringspaces=false,         % underline spaces within strings
showtabs=false,                 % show tabs within strings adding particular underscores
tabsize=2,                      % sets default tabsize to 2 spaces
captionpos=b,                   % sets the caption-position to bottom
breaklines=true,                % sets automatic line breaking
breakatwhitespace=false,        % sets if automatic breaks should only happen at whitespace
escapeinside={\%*}{*)},         % if you want to add a comment within your code
frame=single,                   % adds a frame around the code
}

\autor{Christian Hartung}
\titulo{Detec��o de Vulnerabilidades XPath Injection em Web Services}
\comentario{Trabalho de Gradua��o Interdisciplinar apresentado na Faculdade de Tecnologia - FT como requisito de conclus�o do curso de Tecnologia em An�lise e Desenvolvimento de Sistemas.}
\instituicao{Universidade Estadual de Campinas - UNICAMP\par
 Faculdade de Tecnologia - FT}
\orientador[Orientadora:]{\small Prof. Tania Basso}
\coorientador[Co-orientadora:]{\small Prof. Dr. Regina L. O. Moraes}
\local{Limeira}
\data{2011}

\begin{document}
\capa
\folhaderosto

% Errata

% Dedicat�ria

% Agradecimentos

% Ep�grafe

\sumario

\listoffigures

\listoftables

\pretextualchapter{Lista de siglas, abreviaturas e s�mbolos}
\begin{description*}
  \item[REST] \textit{Representational State Transfer} - Transfer�ncia de Estado Representacional
  \item[HTTP] \textit{Hypertext Transfer Protocol} - Protocolo de Transferencia de Hipertexto
  \item[API] \textit{Application Programming Interface} - Interface de Programa��o de Aplica��es
  \item[RPC] \textit{Remote Procedure Call} - Chamada Remota de Procedimento
  \item[SOAP] \textit{Simple Object Access Protocol} - Protocolo Simples de Acesso a Objetos
  \item[XML] \textit{Extensible Markup Language} - Linguagem de Marca��o Extens�vel
  \item[WSDL] \textit{Web Services Description Language} - Linguagem de Descri��o de Servi�os Web
\end{description*}

\begin{resumo}
 Servi�os web podem apresentar vulnerabilidades de seguran�a. Scanners de vulnerabilidades s�o ferramentas capazes de detectar vulnerabilidades de seguran�a em servi�os web. O objetivo do trabalho � avaliar a seguran�a de servi�os web em rela��o � vulnerabilidade do tipo \emph{XPath Injection} e tamb�m avaliar a efic�cia de scanners de vulnerabilidades na detec��o dessas vulnerabilidades.
\end{resumo}

\chapter{Web Services}
Nos �ltimos anos, muito se tem falado sobre as vantagens de se utilizar a chamada cloud computing (computa��o em nuvem) como arquitetura de TI.

Essa arquitetura permite que as aplica��es fiquem armazenadas em centros de armazenamento de terceiros\cite{Silva-Fabio-Rodrigues:2010:CloudComputing}, e sejam acessadas das m�quinas clientes, reduzindo assim os custos com compra e manuten��o de servidores, dispositivos para armazenamento, backups, etc. Apenas o acesso ao servi�o � pago.

De forma resumida, ao inv�s de instalar um editor de textos na m�quina, ele funciona como um servi�o armazenado no provedor e existe uma interface para acess�-lo. Todos os recursos de hardware necess�rios para rod�-lo, o espa�o para armazenamento de documentos, etc. ficam no provedor e � tudo feito via navegador de internet.

No lado pessoal, sites como Twitter e Facebook vem ganhando muita popularidade, principalmente pelo fato de eles serem acessados atrav�s de aplicativos rodando nos mais variados dispositivos, como desktops, workstations, celulares, consoles de v�deo-game, etc.

Nessas duas situa��es, assim como em diversas outras onde � necess�ria comunica��o entre sistemas separados, faz-se o uso de um conceito chamado web service (servi�o web).

Um web service � um software que implementa diversas tecnologias para transfer�ncia e representa��o de dados que, quando combinadas, que permitem a comunica��o entre os diversos dispositivos conectados em uma rede.

\section{Funcionamento}
Pode-se fazer uma analogia entre um web service e uma sala de aula. Durante uma aula, os alunos tiram que ficam com d�vidas fazem perguntas ao professor, que as responde. Se transformarmos os atores dessa situa��o em sistemas computacionais, com o professor se tornando um servidor e os alunos se tornando os clientes, teremos o funcionamento b�sico de um web service.

\begin{figure}[H]
 \centering
 \includegraphics[width=7cm]{img/Requisicao}%
 \caption{Cliente fazendo requisi��o a um web service}%
 \label{fig:Requisicao}%
\end{figure}

A figura \ref{fig:Requisicao} mostra como funciona um web service. O cliente inicia a comunica��o fazendo uma requisi��o ao servidor, que d� a resposta.

Antes de come�ar a implementa��o de um web service, � preciso definir as tecnologias que ser�o utilizadas para representar os dados e para transferi-los.

\subsection{Representa��o dos dados}
A tecnologia para representa��o dos dados � o idioma que nosso web service fala. De nada adianta nosso professor falar latim se nossos alunos entendem apenas portugu�s. Da mesma forma, de nada adianta o servidor enviar os dados utilizando uma representa��o que os clientes n�o s�o capazes de interpretar.

N�o existe nenhuma norma que obrigue a utilizar determinada representa��o para desenvolvimento de um web service, mas o XML � a representa��o mais utilizada\cite{Costa-Guilherme:2009:SOA}.

\subsection{Transfer�ncia das informa��es}
Al�m de falarem o mesmo idioma, a forma como a informa��o � transmitida tamb�m deve ser compat�vel entre o cliente e o servidor. A figura \ref{fig:ProblemaComunicacao} mostra uma sala onde existe um aluno cego, por�m o professor tenta tirar uma d�vida utilizando a lousa.

\begin{figure}[H]
 \centering
 \includegraphics[width=7cm]{img/ProblemaComunicacao}%
 \caption{Servidor utilizando uma forma de comunica��o incompat�vel}%
 \label{fig:ProblemaComunicacao}%
\end{figure}

Da mesma forma, de nada adianta o servidor responder �s requisi��es de uma forma que o cliente n�o possa interpretar, nem o cliente fazer as requisi��es de uma forma que o servidor n�o entenda.

\section{Vantagens e Desvantagens}
Utilizando web services, � poss�vel dividir o processamento que uma aplica��o precisa realizar entre os clientes e o servidor, tornando os aplicativos mais leves entre ambas as partes.

Por serem baseados em padr�es abertos, a grande maioria das linguagens de programa��o possui suas API's para cria��o de web services, e estas s�o compat�veis entre si, portanto o cliente e o servidor podem ser escritos em linguagens diferentes.

Um problema dessas tecnologias � que o XML pode ser considerado um formato pesado para troca de informa��es, j� que existe um conjunto de regras que devem ser seguidas, principalmente para garantir a compatibilidade.

Outra preocupa��o que se deve ter � com como os dados trafegar�o. Como o uso de web services envolve rede, ou mesmo a internet, � preciso que as requisi��es sejam criptografadas para evitar que os dados sejam obtidos por outros.
\chapter{XPath e XPath Injection}
Uma das aplica��es mais comuns do XML � o armazenamento de dados de forma estruturada e leg�vel, tanto para m�quinas como para humanos.

J� existem tecnologias que permitem pegar os dados armazenados em um documento e convert�-los para outra representa��o, como uma p�gina web ou um documento PDF, tecnologias que permitem que um documento XML referencie outro, tecnologias que garantam a autenticidade de um documento XML, como uma assinatura digital, e tecnologias que ajudam a navegar pelo documento XML. Esta �ltima � conhecida como XPath.

\section{Entendendo o XPath}
Em uma consulta XPath, o documento XML � visto como uma �rvore, onde cada tag, cada atributo, cada texto, cada coment�rio, etc. � tratado como um n�. A linguagem XPath permite navegar por essa �rvore, selecionando um conjunto de n�s baseado em uma s�rie de crit�rios. Desde 1999, XPath � uma recomenda��o da W3C\cite{W3Schools:XPath}.

As principais express�es no XPath s�o:
\begin{table}
 \caption{Principais express�es XPath}
 \label{tab:expressoes-xpath}
 
 \begin{tabular}{lp{11cm}}
  \toprule
  \textbf{Express�o} & \textbf{Descri��o} \\
  \midrule
  \textit{nome} & Seleciona todos os filhos com o nome dado \\
  \midrule
  / & Seleciona a partir do n� atual \\
  \midrule
  // & Seleciona todos os n�s do documento que satisfa�am o crit�rio, partindo do atual \\
  \midrule
  . & Seleciona o n� atual \\
  \midrule
  .. & Seleciona o pai do n� atual \\
  \midrule
  @ & Seleciona um atributo \\
  \bottomrule
 \end{tabular}
\end{table}

O seguinte documento XML � utilizado para verificar os valores m�ximos de cota��es de compra que um usu�rio pode aprovar:
\begin{lstlisting}[language=XML]
<?xml version="1.0" encoding="iso-8859-1"?>
<VALCOMPRAS>
  <VALOR VALUE="500" NUMCOT="1">
    <USERS>
      <USER NAME="EMILY"/>
    </USERS>
  </VALOR>
  <VALOR VALUE="999" NUMCOT="1">
    <USERS>
      <USER NAME="TANIA"/>
    </USERS>
  </VALOR>
  <VALOR VALUE="1000000" NUMCOT="1">
    <USERS>
      <USER NAME="CHRISTIAN"/>
    </USERS>
  </VALOR>
</VALCOMPRAS>
\end{lstlisting}

A est�o algumas consultas XPath aplicadas nesse documento, e se seus resultados:
\begin{description*}
  \item[VALCOMPRAS] Seleciona todos os filhos do elemento raiz (VALCOMPRAS)
  \item[/VALCOMPRAS] Seleciona o elemento raiz
  \item[VALCOMPRAS/VALOR] Seleciona todos os elementos \texttt{VALOR} filhos diretos de \texttt{VALCOMPRAS}
  \item[VALCOMPRAS//USER] Seleciona todos os elementos \texttt{USER} descendentes de \texttt{VALCOMPRAS}. N�o � necess�rio ser um filho direto
  \item[//@VALUE] Seleciona todos os atributos \texttt{VALUE} do documento 
\end{description*}

Para filtrar n�s, faz-se o uso de predicados, express�es entre colchetes. A seguir est�o alguns exemplos de consultas com predicados e seus resultados:
\begin{description*}
	\item[VALCOMPRAS/VALOR[1]]\\
	  Seleciona o primeiro elemento \texttt{VALOR} filho de \texttt{VALCOMPRAS}
	\item[VALCOMPRAS/VALOR[last()]]\\
	  Seleciona o �ltimo elemento \texttt{VALOR} filho de \texttt{VALCOMPRAS}
	\item[VALCOMPRAS/VALOR[position() <= 2]]\\
	  Seleciona os dois primeiros elementos \texttt{VALOR} filhos de \texttt{VALCOMPRAS}
  \item[//VALOR[@VALUE]]\\
    Seleciona todos os elementos \texttt{VALOR} que possuam um atributo \texttt{VALUE}
  \item[//USERS[USER]]\\
    Seleciona todos os elementos \texttt{USERS} que tenham uma tag \texttt{USER} como filho
  \item[VALCOMPRAS/VALOR[@VALUE >= 1000]//USER]]\\
    Seleciona todos os elementos \texttt{USER} que descendentes de um elemento \texttt{VALOR} com o atributo VALUE maior ou igual a 1000 
\end{description*}

\section{XPath Injection}
Al�m de transferir dados em formato XML, muitos desenvolvedores de web services optam por armazenar alguns dados internamente em documentos XML. Utilizando este formato, ao receber uma requisi��o, � realizada uma consulta Xpath sobre esses documentos utilizando os par�metros passados, retornando obtendo a resposta desejada.

O c�digo a seguir, escrito em C\# mostra esse comportamento em um web service que consulta um documento XML parecido com o do exemplo anterior:
\begin{lstlisting}[language=C++]
XmlDocument doc = new XmlDocument();
doc.Load(ConfigurationManager.AppSettings["aprovval.xml"]);
var n= doc.SelectNodes("//VALOR[@VALUE >=" + valor + "]/USERS/USER/@NAME");

var sb = new StringBuilder();

for (int i = 0; i < n.Count; i++)
{
  string s = n.Item(i).Value + ';';
  if(!sb.ToString().Contains(s))
  sb.Append(s);
}
ViewBag.Usuarios = sb.ToString();
return View();
\end{lstlisting}

Na linha 3, � realizada uma consulta XPath baseada em um par�metro para o Web Service. A ideia dessa consulta � obter todos os nomes de usu�rio que podem aprovar cota��es de determinado valor. Supondo que a cota��o de compras seja de R\$1000, essa consulta ficaria como:
\begin{lstlisting}[numbers=none]
//VALOR[@VALUE >= 1000]/USERS/USER/@NAME
\end{lstlisting}
o que retorna a lista de nomes corretamente.

Se este par�metro \texttt{valor} tiver o formato de uma consulta XPath, � poss�vel retornar usu�rios que n�o tenham permiss�o para aprovar este valor. Por exemplo, se o par�metro tiver o formato \texttt{1000 | USERS}, a consulta ficaria como:
\begin{lstlisting}[numbers=none]
 //VALOR[@VALUE >= 1000 | USERS]/USERS/USER/@NAME
\end{lstlisting}
retornando os nomes de todos os usu�rios no documento, independente do valor permitido. Da mesma forma, se a consulta realizada tiver o formado
\begin{lstlisting}[numbers=none]
 //VALOR[@VALUE >= 1000 | //@NAME="EMILY"]/USERS/USER/@NAME
\end{lstlisting}
o nome do usu�rio EMILY ser� retornado independente do valor da cota��o.

A esse tipo de manipula��o da consulta realizada no documento XML � dado o nome \emph{XPath Injection}.

\bibliography{bibliografia}
\end{document}