% Trabalho de Gradua��o Interdisciplinar - Christian Hartung
\documentclass[times]{abnt}

\usepackage[brazil]{babel}
\usepackage[latin1]{inputenc}
\usepackage[T1]{fontenc}
\usepackage{color}
\usepackage{graphics}
\usepackage[pdftex]{graphicx}
\usepackage{float}
\usepackage{booktabs}
\usepackage{mdwlist}
\usepackage{booktabs}
\usepackage[num]{abntcite}
\usepackage{url}

% A grossura padr�o da linha de assinatura � 0. Aumento para 0.4, s� para ela aparecer
\setlength{\ABNTsignthickness}{0.4pt}
\setlength{\ABNTsignskip}{2cm}
\setlength{\ABNTsignwidth}{6.5cm}

%%%%%%%%%%%%%%%%%%%% Comandos utilizados

% Texto em outro idioma
\newcommand{\estrangeiro}[1]{\textit{#1}}

% Insere uma figura
\newcommand{\figura}[3]{
 \begin{figure}[h]
  \centering
  \includegraphics[width=#3]{img/#1}%
  \caption{#2}%
  \label{fig:#1}%
 \end{figure}
}


\autor{Christian Hartung}
\titulo{Uma avalia��o de Scanners de Vulnerabilidade na detec��o de XPath Injection em Aplica��es Web}
\comentario{Trabalho de Gradua��o Interdisciplinar apresentado na Faculdade de Tecnologia - FT como requisito de conclus�o do curso de Tecnologia em An�lise e Desenvolvimento de Sistemas.}
\instituicao{Universidade Estadual de Campinas - UNICAMP\par
 Faculdade de Tecnologia - FT}
\orientador[Orientadora:]{\small Prof. Tania Basso}
\coorientador[Co-orientadora:]{\small Prof. Dr. Regina L. O. Moraes}
\local{Limeira}
\data{2011}

\begin{document}
\capa
\folhaderosto

% Errata

% Dedicat�ria

% Agradecimentos

% Ep�grafe

\sumario

\listoffigures

\listoftables

\begin{resumo}
 Aplica��es web podem apresentar brechas de seguran�a. Scanners de vulnerabilidades s�o ferramentas capazes de identificar essas brechas. O objetivo deste trabalho � avaliar a seguran�a de aplica��es web em rela��o � vulnerabilidade do tipo \emph{XPath Injection} e tamb�m avaliar a efic�cia de scanners de vulnerabilidades na detec��o dessas vulnerabilidades.
\end{resumo}

\begin{abstract}
 Web application can have security holes. Web Application vulnerability scanners are tools capable of identifying those holes. This work tries to evaluate the security of web application with respect of XPath Injection vulnerability and also assess the effectiveness of vulnerability scanners to detect these vulnerabilities. 
\end{abstract}

\chapter{Introdu��o}
Cada vez mais a sociedade moderna est� dependente dos computadores para realiza��o de suas atividades cotidianas. Com isso a internet se tornou um item essencial para as pessoas.

S�o compras realizadas e contas pagas sem a necessidade de sair de casa. S� � necess�rio um computador e uma conex�o com a internet.

Esses servi�os que facilitam tanto a vida, podem acumular informa��es sobre os usu�rios, como, por exemplo, localiza��o, dados financeiros, e outros dados pessoais ou informa��es confidenciais. Os usu�rios acreditam que essas informa��es est�o bem protegidas, mas isso nem sempre � verdade.

Tome como exemplo a \emph{PSN} (\estrangeiro{PlayStation Network}), a rede utilizada pelo console de v�deo-game \estrangeiro{PlayStation 3} para organizar partidas \estrangeiro{online}, para comprar e baixar jogos, realizar comunica��o entre os usu�rios, dentre outros recursos. Recentemente uma vulnerabilidade foi encontrada\cite{info:PSN} nessa rede e v�rios dados de usu�rios foram obtidos por quem a explorou. Entre esses dados estavam informa��es sobre os cart�es de cr�dito utilizados para fazer compras na PSN.

O caso da PSN mostra que mesmo os maiores servi�os existentes est�o sujeitos a todo tipo de vulnerabilidade e nem sempre ferramentas tradicionais como \estrangeiro{firewalls} e criptografia s�o suficientes para proteg�-las. Al�m disso, o objetivo dessas ferramentas tradicionais � proteger a rede. Elas n�o mitigam ataques a aplica��es Web e, dessa forma, \estrangeiro{hackers} est�o tamb�m com foco nessas aplica��es, onde vulnerabilidades de seguran�a criadas, na maioria das vezes, por codifica��o n�o segura, representam maior risco\cite{dsn:2008:mapping_software_faults}.

Um exemplo desse tipo de vulnerabilidade ocorreu no \emph{Twitter} em setembro de 2010\cite{info:Twitter}. Apesar de os servidores do Twitter estarem devidamente protegidos, um erro na aplica��o permitiu que c�digo malicioso fosse executado na p�gina, fazendo com que janelas \estrangeiro{pop-up} fossem abertas ao passar o \estrangeiro{mouse} sobre um \estrangeiro{link}.

\estrangeiro{Scanners} de vulnerabilidade s�o ferramentas que fazem a an�lise das aplica��es \estrangeiro{web} de forma autom�tica com o objetivo de detectar e reportar vulnerabilidades de seguran�a\cite{Tania:2010:Scanners}. Para utilizar um \estrangeiro{scanner} de vulnerabilidade, � essecial confiar nos resultados oferecidos por ele, portanto a efic�cia na detec��o das vulnerabilidades desse \estrangeiro{scanner} deve ser avaliada. Essa avalia��o � feita a partir de dois aspectos: falta de cobertura (quando o \estrangeiro{scanner} deixa de detectar vulnerabilidades que realmente existem na aplica��o) e falsos positivos (o \estrangeiro{scanner} detecta vulnerabilidades que n�o existem na aplica��o). Falsos positivos implicam desperd�cio de tempo para corre��es que de fato n�o s�o necess�rias.

O objetivo deste trabalho � analisar a efic�cia de alguns desses \estrangeiro{scanners} de vulnerabilidade para encontrar um tipo de vulnerabilidade espec�fico: \estrangeiro{XPath Injection}.

Este trabalho est� organizado da seguinte forma:
\begin{description*}
	\item[Cap�tulo \ref{chap:WebApps}:] introduz aplica��es \estrangeiro{web} e servi�os \estrangeiro{web}, as duas tecnologias que esses \estrangeiro{scanners} s�o capazes de analisar;
	\item[Cap�tulo \ref{chap:Vulnerabilidades}:] apresenta estat�sticas sobre as principais vulnerabilidades exploradas em aplica��es \estrangeiro{web} atualmente;
	\item[Cap�tulo \ref{chap:XPath}:] mostra conceitos de XML e XPath e descreve as vulnerabilidades do tipo \estrangeiro{XPath Injection};
	\item[Cap�tulo \ref{chap:Estudo}:] apresenta o estudo realizado e os resultados obtidos;
	\item[Cap�tulo \ref{chap:Conclusao}:] conclus�es e ideias para estudos futuros;
\end{description*}
\chapter{Aplica��es e Servi�os Web} \label{chap:WebApps}
Nos �ltimos anos, muito se tem falado sobre as vantagens de se utilizar a chamada \emph{cloud computing} (computa��o em nuvem) como arquitetura de TI (Tecnologia da Informa��o) \cite{Cloud:computerworld} \cite{Cloud:idgnow} \cite{Cloud:info:ml} \cite{Cloud:info:entrevista} \cite{Cloud:olhardigital}. Essa arquitetura permite que as aplica��es sejam armazenadas em centros de armazenamento de terceiros\cite{Silva-Fabio-Rodrigues:2010:CloudComputing} e utilizadas por meio de m�quinas clientes, reduzindo, assim, os custos com aquisi��o e manuten��o de servidores, dispositivos para armazenamento, \estrangeiro{backups}, etc. O �nico requisito � que o cliente tenha um navegador de internet instalado.

A vantagem da utiliza��o da computa��o em nuvem est� no fato de o usu�rio, ao inv�s de instalar, por exemplo, um editor de textos, uma planilha eletr�nica ou um editor apresenta��es em seu computador, poder utilizar esses softwares como um servi�o armazenado em um provedor. Todos os recursos de hardware necess�rios para executar esses softwares, o espa�o para armazenamento de documentos, e outros recursos s�o responsabilidade desse provedor. O usu�rio precisa apenas utilizar o servi�o por meio do navegador de internet. Aplica��es com essa caracter�stica de serem acessadas pelo navegador s�o conhecidas como \emph{aplica��es web}.

Mas nem sempre uma aplica��o \estrangeiro{web} � poss�vel ou vantajoso. Se o servi�o em quest�o precisa ser acessado de um \estrangeiro{smartphone}, onde a experiencia com o navegador � mais limitada, pelo tamanho da tela ser menor, o poder de processamento reduzido, etc., ou o usu�rio n�o possui uma conex�o r�pida o bastante para acessar o servi�o, ou ainda o usu�rio prefere ter um aplicativo instalado para utilizar o servi�o, utilizar a aplica��o \estrangeiro{web} diretamente se tornaria um problema.

Nesses casos, o provedor do servi�o pode disponibilizar uma API (\estrangeiro{Application Programming Interface}, ou Interface de Programa��o de Aplica��es), para que seja desenvolvido um aplicativo que utilize o servi�o em quest�o. Essa API � chamada \emph{web service} (servi�o \estrangeiro{web}).

As se��es seguintes apresentam uma breve introdu��o sobre essas duas tecnologias (aplica��es \estrangeiro{web} e \estrangeiro{web services}): como funcionam, suas vantagens e desvantagens.

%%%%%%%%%%%%%%%%%%%%%%%%%%%%%%%%%%%%%%%%%%%%%%%%%%%%%%%%%%
\section{Aplica��es \estrangeiro{web}}
Uma aplica��o \estrangeiro{web} � qualquer aplica��o que utilize um navegador de internet como cliente\cite{About:WebApps}. Apesar de o termo \estrangeiro{cloud computing} ter surgido em meados de 2006\cite{zdnet:IntroCloud}, as primeiras aplica��es \estrangeiro{web} surgiram por volta de 1987, antes mesmo de a internet se popularizar\cite{About:InternetHistory} 

Aplica��es \estrangeiro{web} funcionam utilizando uma arquitetura cliente-servidor. Nessa arquitetura, o cliente (navegador de internet) solicita um recurso do servidor. O servidor, por sua vez, responde com uma representa��o do recurso solicitado em um formato que o navegador pode entender: a HTML (\estrangeiro{Hypertext Markup Language}),  Linguagem de Marca��o de Hipertexto. 

A HTML sozinha � uma linguagem limitada\cite{About:InternetHistory}. Tudo que ela faz � dizer ao navegador como interpretar o conte�do da p�gina e ent�o, apenas p�ginas est�ticas podem ser constru�das com ela. Devido a essa limita��o da HTML, as primeiras aplica��es \estrangeiro{web} n�o passavam de p�ginas exibidas em determinada sequ�ncia, baseadas em dados de entrada do usu�rio. Com o tempo, novas tecnologias surgiram para permitir que mais pudesse ser feito com a HTML. A principal delas surgiu em 1995: as linguagens \estrangeiro{client-side}.

Linguagens \estrangeiro{client-side} s�o linguagens de programa��o que s�o executadas no lado cliente da aplica��o, ou seja, no navegador de internet. Elas permitiram que alguma l�gica fosse adicionada �s p�ginas HTML, tornando as aplica��es Web mais interativas. Atualmente a principal linguagem \estrangeiro{client-side} � o JavaScript, j� que � a linguagem de script padr�o do HTML5\footnote{O HTML5 � a vers�o do HTML em desenvolvimento}\cite{whatwg:HTML}.

Com o uso de JavaScript, as aplica��es \estrangeiro{web} t�m se tornado bastante din�micas e j� se assemelham �s aplica��es tradicionais (aplicativos que s�o executados no \estrangeiro{desktop}), algumas at� substituindo essas aplica��es tradicionais. Atualmente, milh�es de pessoas utilizam aplica��es \estrangeiro{web} quando acessam o \estrangeiro{webmail}, o bate-papo, compras online, \estrangeiro{internet banking}, etc

\subsection{Arquitetura de uma Aplica��o \estrangeiro{web}}
Uma aplica��o \estrangeiro{web} utiliza uma arquitetura multicamadas. As principais camadas s�o a camada de apresenta��o, camada de neg�cios e camada de persist�ncia.

\subsubsection{Camada de Apresenta��o}
A \emph{camada de apresenta��o} � a interface da aplica��o apresentada aos usu�rios. � a parte da aplica��o exibida no navegador \estrangeiro{web}, com a qual o usu�rio pode interagir.

\subsubsection{Camada de Neg�cios}
� a camada respons�vel pela l�gica de neg�cios da aplica��o, intermedi�ria entre a as camadas de apresenta��o e de persist�ncia, processando os dados que o usu�rio informa e passando-os para a camada de persist�ncia, ou lendo os dados que o usu�rio solicitou e exibindo-os na camada de apresenta��o.

Essa camada normalmente � formada por um servidor HTTP (\estrangeiro{Hypertext Transfer Protocol}, - Protocolo de Transfer�ncia de Hipertexto), ou servidor \estrangeiro{web}, como � mais conhecido, que recebe as requisi��es dos clientes e respondem enviando representa��es do que o cliente pediu. Essas representa��es podem ser imagens, v�deos, documentos HTML, etc.

O HTTP � o protocolo padr�o para transfer�ncia de dados pela internet e est� atualmente na vers�o 1.1\cite{RFC:2616}. A figura \ref{fig:TemplateHTTP} mostra o formato de uma requisi��o HTTP.

\figura{TemplateHTTP}{Formato de uma requisi��o HTTP}{7cm}

Na Figura \ref{fig:TemplateHTTP}, ``\estrangeiro{Host}'', ``\estrangeiro{Cookie}'' e ``\estrangeiro{From}'' s�o os cabe�alhos da requisi��o.  Uma requisi��o pode ter qualquer quantidade de cabe�alhos, padronizados ou personalizados.  Eles s�o utilizados para enviar meta-informa��es com cada requisi��o, ou seja, enviar informa��es adicionais sobre o recurso que est� sendo transmitido. Exemplos de cabe�alhos comumente usados s�o:
\begin{description*}
	\item[Host:] identifica o dom�nio que est� sendo acessado;
	\item[From:] cont�m o \estrangeiro{e-mail} do usu�rio que fez a requisi��o;
	\item[Cookie:] cont�m informa��es de autentica��o na aplica��o em quest�o;
	\item[Content-Type:] informa o tipo de dados em que o conte�do da requisi��o est� codificado;
	\item[Content-Length:] informa o tamanho do conte�do da requisi��o;
\end{description*}

O \emph{m�todo} utilizado na requisi��o indica o que o servidor deve fazer com o recurso transmitido. Dois m�todos principais s�o utilizados em uma aplica��o web: \emph{GET} e \emph{POST}.

O cliente utiliza o m�todo \texttt{GET} quando deseja obter recursos do servidor. A figura \ref{fig:HTTP-GET} mostra um exemplo de requisi��o \texttt{GET} para obter o recurso \texttt{usuarios/chrishartung} do servidor respons�vel pelo dom�nio \texttt{exemplo.com}. A p�gina acessada para isso � \url{http://exemplo.com/usuarios/chrishartung}.

\figura{HTTP-GET}{Requisi��o HTTP GET}{5cm}

J� o m�todo \texttt{POST} � utilizado quando o cliente deseja enviar informa��es para o servidor. Um uso comum dessa requisi��o na internet � para o preenchimento de formul�rios. A figura \ref{fig:HTTP-POST} mostra um exemplo de requisi��o \texttt{POST} que altera a institui��o do usu�rio \texttt{chrishartung}. Observe que neste caso foi informado o tipo dos dados em que o conte�do foi codificado (\emph{application/x-www-form-urlencoded}) e o tamanho da mensagem. 

\figura{HTTP-POST}{Requisi��o HTTP POST}{11cm}

O tipo \texttt{application/x-www-form-urlencoded} � utilizado como padr�o no HTML\cite{W3C:HTML4} para envio de formul�rios. Neste formato, apenas caracteres ASCII\footnote{O \estrangeiro{American Standard Code for Information Interchange} � uma codifica��o de caracteres baseado no alfabeto ingl�s. Consiste de 128 caracteres, sendo 94 vis�veis, 33 caracteres de controle (caracteres que informam como o texto deve ser processado) e o caractere de espa�o.} podem ser utilizados, ent�o caracteres n�o-ASCII devem ser codificados no formato $\%HH$, onde \texttt{HH} s�o dois d�gitos hexadecimais, e espa�os s�o substitu�dos por $+$. Os campos s�o enviados como pares \texttt{nome=valor}, separados pelo caractere \texttt{\&}.

O servidor \estrangeiro{web} sozinho s� � capaz de enviar conte�do est�tico para o cliente. O lado servidor da aplica��o (\estrangeiro{server-side}) tamb�m entende alguma linguagem de programa��o. Essa linguagem de programa��o � respons�vel por construir a resposta para a requisi��o dinamicamente, ler e  gravar informa��es no servidor de dados, etc.

Exemplos comuns de linguagens rodando no servidor s�o o PHP (\estrangeiro{Hypertext Preprocessor}), ASP (\estrangeiro{Active Server Pages}), JSP (\estrangeiro{Java Server Pages}), ColdFusion, etc.

\subsubsection{Camada de Persist�ncia}
A maioria das aplica��es \estrangeiro{web} lida com armazenamento de informa��es, tempor�ria ou permanentemente. Isso pode ser feito utilizando Sistemas Gerenciadores de Banco de Dados (SGBD's) ou outros sistemas mais simples, que podem chegar a arquivos de texto. A camada de persist�ncia � respons�vel por armazenar essas informa��es.

Servidores de dados que comp�em a camada de persist�ncia s�o os computadores mais protegidos da empresa, j� que armazenam os dados de todos os usu�rios da aplica��o. A configura��o desses servidores tem um impacto muito grande no desempenho da aplica��o. Essas s�o as m�quinas que possuem hardware mais moderno e recebem investimentos com maior frequ�ncia.

\subsection{Vantagens das aplica��es Web}
Aplica��es \estrangeiro{web} v�m ganhando muita aten��o tanto em ambientes corporativos como no uso pessoal. Isso porque oferecem algumas vantagens sobre as aplica��es de \textit{desktop}\cite{AJAX-PHP-WebApps}:
\begin{description*}
	\item[F�cil distribui��o:] Basta um navegador e uma conex�o com a internet ou intranet. N�o � necess�rio que o departamento de TI instale a aplica��o em cada esta��o da empresa;
	\item[Independente de plataforma:] O usu�rio final pode usar Windows, Mac, Linux, Solaris, iPhone, etc. como sistema operacional e qualquer navegador dispon�vel\footnote{Se a aplica��o foi constru�da utilizando os padr�es da W3C corretamente}.
	\item[Dados centralizados:] N�o h� dados salvos na m�quina do usu�rio. Tudo est� no servidor.
\end{description*}

Alguns problemas tamb�m surgem ao utilizar essa arquitetura. Os principais deles s�o:
\begin{description*}
	\item[Depend�ncia da rede:] Todos os usu�rios do sistema dependem da rede ou internet para utiliz�-lo. Se ocorrer algum problema com o link da internet utilizado, todos os usu�rios ficar�o comprometidos.
	\item[Depend�ncia do servidor:] Se o servidor ficar indispon�vel, o sistema todo tamb�m fica.
\end{description*}

%%%%%%%%%%%%%%%%%%%%%%%%%%%%%%%%%%%%%%%%%%%%%%%%%%%%%%%%%%%%%%%%%%%%%%%
\section{\estrangeiro{Web services}}
Um \emph{web service} (servi�o web) � um software que implementa diversas tecnologias para transfer�ncia e representa��o de dados que, quando combinadas, permitem a comunica��o entre os diversos dispositivos conectados em uma rede\cite{W3C:ws-arch}.

Diferentemente das aplica��es \estrangeiro{web}, onde �nico cliente � o navegador de internet, um \estrangeiro{web service} pode ter os mais variados tipos de clientes, como aplicativos pr�prios, aplicativos de terceiros, linha de comando, celulares e at� mesmo aplica��es \estrangeiro{web} ou outros \estrangeiro{web services}.

Um \estrangeiro{web service} � um software que fica no servidor e que responde a requisi��es dos clientes.

\figura{WebServiceLoja}{Exemplo de aplica��o web utilizando web services}{10cm}

A figura \ref{fig:WebServiceLoja} mostra como funciona uma aplica��o \estrangeiro{web} de cota��o de pre�os que consome os servi�os oferecidos pelos diversos \estrangeiro{web services} das lojas virtuais. A \emph{Aplica��o Cliente} � o navegador de internet rodando no computador do usu�rio. Quando o usu�rio consulta o pre�o de um produto, a aplica��o \estrangeiro{web} \emph{Sistema de Pesquisa} consulta nos \estrangeiro{web services} das lojas virtuais a cota��o deste produto. Com as respostas obtidas dos \estrangeiro{web services}, o Sistema de Pesquisa monta a p�gina com as informa��es que o cliente deseja.

Antes de come�ar a implementa��o de um \estrangeiro{web service}, � preciso definir as tecnologias que ser�o utilizadas para representar os dados e para transferi-los.

\subsection{Representa��o dos dados}
A tecnologia para representa��o dos dados � o idioma que o web service fala. De nada adianta um professor falar latim se seus alunos entendem apenas portugu�s. Da mesma forma, de nada adianta o servidor enviar os dados utilizando uma representa��o que os clientes n�o s�o capazes de interpretar.

N�o existe nenhuma norma que obrigue a utilizar determinada representa��o para desenvolvimento de um \estrangeiro{web service}\footnote{A natureza do SOAP induz � utiliza��o de XML, mas n�o � obrigat�rio usar SOAP para cria��o de um \estrangeiro{web service}.}. Atualmente os formatos mais utilizados s�o o XML (\estrangeiro{Extensible Markup Language}, ou Linguagem de Marca��o Extens�vel) e o JSON (\estrangeiro{JavaScript Object Notation}, ou Nota��o de Objeto JavaScript).

\subsubsection{XML}
A XML � uma linguagem de marca��o padronizada pela W3C\cite{W3C:XML}, que a define como um formato para representar informa��es estruturadas na forma de texto. Sua ado��o se d� principalmente pelo fato de ter sido criada pensando em internet, em facilidade de uso, em ser independente de plataforma ou linguagem de programa��o e em ser leg�vel para m�quinas e humanos.

\figura{XMLUsuarios}{Exemplo de documento XML}{7cm}

A figura \ref{fig:XMLUsuarios} mostra um exemplo de documento XML que representa um usu�rio do sistema. A primeira linha cont�m a declara��o XML, que identifica a vers�o da especifica��o da XML sendo utilizada (1.0) e o conjunto de caracteres utilizado (UTF-8). Na segunda linha est� o elemento raiz do documento (``usuarios''). Os documentos XML tamb�m podem ser representados em estrutura de �rvore\cite{W3Schools:XML}, que tem esse elemento como raiz. As linhas seguintes definem elementos filhos da raiz, ou at� mesmo filhos dos filhos da raiz. A �ltima linha encerra o elemento raiz, encerrando o documento. A figura \ref{fig:XMLUsuariosArvore} mostra esse mesmo documento em formato de �rvore.

\figura{XMLUsuariosArvore}{Representa��o de documento XML em �rvore}{7cm}

Os documentos XML s�o formados por duas constru��es principais: as \estrangeiro{tags} (marcadores) e seus atributos. Uma \estrangeiro{tag} � um elemento entre os sinais de ``<'' e ``>'' e os atributos s�o pares nome/valor que ficam dentro da \estrangeiro{tag}. No exemplo da Figura \ref{fig:XMLUsuarios}, ``usuarios'', ``usuario'', ``nome'' e ``nascimento'' s�o \estrangeiro{tags}, enquanto ``login'' � um atributo da \estrangeiro{tag} ``usuario''. A especifica��o do XML n�o define nenhuma \estrangeiro{tag} ou atributo, cabendo ao desenvolvedor do sistema defin�-los.

\subsubsection{JSON}
Apesar de o XML ser o formato mais utilizado para transfer�ncia de dados\cite{Costa-Guilherme:2009:SOA}, ele � um formato que possui muitas regras, o que o torna pesado.

Pensando nisso o formato JSON foi criado e padronizado\cite{IETF:JSON}, com base na sintaxe para cria��o de objetos da linguagem JavaScript. A figura \ref{fig:JSONUsuarios} mostra como ficaria o documento XML da figura \ref{fig:XMLUsuarios}, que representa um usu�rio do sistema, utilizando o formato JSON.

\figura{JSONUsuarios}{Exemplo de objeto JSON}{5cm}

O JSON � formado por um conjunto de pares nome/valor. O nome � uma \estrangeiro{string} (definido como uma sequ�ncia de caracteres entre aspas) e o valor pode ser qualquer um dos tipos de dados suportados:
\begin{description*}
	\item[string:] cadeias de caracteres entre aspas;
	\item[n�meros:] valores num�ricos, com ou sem casas decimais;
	\item[booleanos:] \estrangeiro{true} (verdadeiro) ou \estrangeiro{false} (falso);
	\item[\estrangeiro{arrays}:] conjunto de valores delimitado por colchetes, como \texttt{[1, [3,4], \{``nome'':``fulano''\}]};
	\item[objetos:] conjunto de pares nome/valor;
	\item[vazio:] nenhum valor, representado pela palavra \texttt{null};
\end{description*}

Estudos indicam que o formato JSON pode ser analisado at� cem vezes mais r�pido que o XML\cite{JSONxXML}.

\subsection{Transfer�ncia das informa��es}
Al�m de falarem o mesmo idioma, a forma como a informa��o � transmitida tamb�m deve ser compat�vel entre o cliente e o servidor.

De nada adianta o cliente fazer requisi��es utilizando um protocolo que o servidor n�o entende, da mesma forma que de nada adianta o servidor responder �s requisi��es de uma forma que o cliente n�o possa interpretar.

Atualmente existem duas formas para transfer�ncia de dados que tem sido difundidas: o \emph{SOAP} (\estrangeiro{Simple Object Access Protocol}, ou Protocolo Simples de Acesso a Objetos) e o \emph{REST} (\estrangeiro{Representational State Transfer}, ou Transfer�ncia de Estado Representacional).

\subsubsection{SOAP}
O SOAP � um protocolo para troca de informa��es por meio de mensagens XML sobre um protocolo de aplica��o, normalmente o HTTP.  � um dos protocolos mais utilizados para cria��o de \estrangeiro{web services}, principalmente pelo fato de ser baseado em XML, o que o torna compat�vel com qualquer linguagem de programa��o, al�m de ser uma recomenda��o da W3C\cite{W3C:SOAP}.

Uma comunica��o utilizando SOAP funciona da seguinte forma: quando o cliente se conecta ao servi�o, obt�m um documento listando os m�todos dispon�veis, seus par�metros e tipos de dados. Este documento � chamado \emph{WSDL} (\estrangeiro{Web Services Description Language}, ou Linguagem de Descri��o de Servi�os Web) , uma linguagem tambem baseada em XML. As figuras \ref{fig:WSDL-Request} e \ref{fig:WSDL-Response} s�o partes de um documento WSDL que define uma mensagem para obter os dados de uma cidade (m�todo \texttt{GetCidade}). Na figura \ref{fig:WSDL-Request} est� o trecho que define a \emph{requisi��o}, ou seja, o formato da mensagem que deve ser enviado para o servidor. A figura \ref{fig:WSDL-Response} mostra o trecho que define o formato da \emph{resposta} que o servidor envia.

\figura{WSDL-Request}{Documento WSDL para um servi�o que obt�m dados de uma cidade - Requisi��o}{7cm}

\figura{WSDL-Response}{Documento WSDL para um servi�o que obt�m dados de uma cidade - Resposta}{8cm}

Tendo este documento, o aplicativo cliente � capaz de fazer a requisi��o SOAP propriamente dita, enviando uma mensagem HTTP semelhante � da figura \ref{fig:SOAP}. Na linha 6 est� a chamada ao m�todo \texttt{GetCidade} e na linha 7 est� sendo passado o par�metro \texttt{Cidade} com um valor informado pelo usu�rio.

\figura{SOAP}{Requisi��o SOAP}{9cm}

\subsubsection{REST}
O REST n�o � um padr�o, mas um conceito. � um conjunto de \estrangeiro{constraints} (restri��es) que, quando seguidas, tornam o servi�o \estrangeiro{RESTful}\cite{REST}.

Diferente do SOAP, que cria um novo protocolo sobre o HTTP para a transfer�ncia da mensagem, em um servi�o \estrangeiro{RESTful} toda mensagem � transmitida diretamente no HTTP\footnote{O conceito foi desenvolvido sobre o protocolo HTTP, mas nada impede que seja utilizado outro protocolo, desde que as \estrangeiro{constraints} possam ser implantadas.}.

O REST trabalha com a transfer�ncia de representa��es de recursos de um lado para o outro da aplica��o, acessadas atrav�s de identificadores. Um recurso � um elemento conceitual da aplica��o. Pode ser um usu�rio, uma cidade, um documento, pode estar armazenado em mem�ria, em um banco de dados, em um documento XML. A representa��o � uma materializa��o desse conceito de uma forma que o aplicativo possa entender, como um documento XML ou JSON. A tabela \ref{tab:elementos-rest} mostra esses elementos e um exemplo deles em um servi�o para controle de usu�rios.

\begin{table}[h]
 \centering
 \begin{tabular}{lp{10cm}}
 \toprule
  \textbf{Elemento} & \textbf{Exemplo} \\
  \midrule
  Recurso & Usu�rio \\
  \midrule
  Identificador & URI (\estrangeiro{Uniform Resource Identifier}, ou Identificador Uniforme de Recursos) onde o recurso pode ser encontrado. \texttt{/usuarios/chrishartung} \\
  \midrule
  Representa��o & Documento JSON representando um usu�rio. \texttt{\{``usuario'':\{``login'':``chrishartung'', ``nome'':``Christian Hartung'', ``nascimento'':``1988-03-08''\}} \\
  \bottomrule
 \end{tabular}
 
 \caption{Elementos de uma aplica��o \estrangeiro{RESTful}}
 \label{tab:elementos-rest}
\end{table}

As \estrangeiro{constraints} que devem ser respeitadas s�o\cite{REST}:
\begin{description*}
	\item[Cliente-servidor:] Deve haver uma clara separa��o entre as responsabilidades do cliente e do servidor. O servidor n�o deve se preocupar com como o recurso ser� exibido, e o cliente n�o deve se preocupar em como o recurso ser� armazenado. Isso permite que as duas partes fiquem mais simples e evoluam de forma independente.
	\item[\estrangeiro{Steteless}:] O servidor n�o deve guardar nenhuma informa��o intermedi�ria sobre o processamento. Toda essa informa��o fica no cliente, e � transmitida na requisi��o sempre que necess�ria.
	\item[\estrangeiro{Cache}:] Deve haver uma marca��o que indica se a resposta pode ser guardada pelo cliente para uso futuro ou n�o.
	\item[Interface uniforme:] Ter uma interface uniforme permite que o sistema fique mais desacoplado e ajuda que cada componente evolua de forma independente. Neste ponto est�o as URI's.
	\item[Multicamadas:] Um cliente nunca pode assumir que est� conectado diretamente ao servidor. Podem existir m�quinas intermedi�rias para aumentar a seguran�a e escalabilidade do servi�o, como \estrangeiro{firewalls}, servidores \estrangeiro{proxy}, etc. Da mesma forma, o servidor n�o pode assumir que est� conectado diretamente ao cliente.
	\item[C�digo sob demanda:] Esta \estrangeiro{constraint} � opcional e diz que o cliente deve ser capaz de executar c�digo enviado pelo servidor.
\end{description*}

O exemplo mais utilizado de servi�o \estrangeiro{RESTful} � a \emph{rede mundial de computadores} (\estrangeiro{World Wide Web}): a internet\cite{REST}.

\subsection{Vantagens e Desvantagens}
Utilizando \estrangeiro{web services}, � poss�vel dividir o processamento que uma aplica��o precisa realizar entre os clientes e o servidor, tornando os aplicativos nas duas partes mais leves.

Por serem baseados em padr�es abertos, a grande maioria das linguagens de programa��o possui suas API's para cria��o de \estrangeiro{web services}, e estas s�o compat�veis entre si, portanto o cliente e o servidor podem ser escritos em linguagens diferentes.

Quando utilizando SOAP, uma limita��o � que os dados s� podem ser transmitidos em XML, dando menor flexibilidade para a aplica��o.

Outra preocupa��o que se deve ter � com como os dados trafegar�o. Como o uso de \estrangeiro{web services} envolve rede, ou mesmo a internet, � preciso que as requisi��es sejam criptografadas para evitar que os dados sejam obtidos por outros.
\chapter{XPath e XPath Injection}
Uma das aplica��es mais comuns do XML � o armazenamento de dados de forma estruturada e leg�vel, tanto para m�quinas como para humanos.

J� existem tecnologias que permitem pegar os dados armazenados em um documento e convert�-los para outra representa��o, como uma p�gina web ou um documento PDF, tecnologias que permitem que um documento XML referencie outro, tecnologias que garantam a autenticidade de um documento XML, como uma assinatura digital, e tecnologias que ajudam a navegar pelo documento XML. Esta �ltima � conhecida como XPath.

\section{Entendendo o XPath}
Em uma consulta XPath, o documento XML � visto como uma �rvore, onde cada tag, cada atributo, cada texto, cada coment�rio, etc. � tratado como um n�. A linguagem XPath permite navegar por essa �rvore, selecionando um conjunto de n�s baseado em uma s�rie de crit�rios. Desde 1999, XPath � uma recomenda��o da W3C\cite{W3Schools:XPath}.

As principais express�es no XPath s�o:
\begin{table}
 \caption{Principais express�es XPath}
 \label{tab:expressoes-xpath}
 
 \begin{tabular}{lp{11cm}}
  \toprule
  \textbf{Express�o} & \textbf{Descri��o} \\
  \midrule
  \textit{nome} & Seleciona todos os filhos com o nome dado \\
  \midrule
  / & Seleciona a partir do n� atual \\
  \midrule
  // & Seleciona todos os n�s do documento que satisfa�am o crit�rio, partindo do atual \\
  \midrule
  . & Seleciona o n� atual \\
  \midrule
  .. & Seleciona o pai do n� atual \\
  \midrule
  @ & Seleciona um atributo \\
  \bottomrule
 \end{tabular}
\end{table}

O seguinte documento XML � utilizado para verificar os valores m�ximos de cota��es de compra que um usu�rio pode aprovar:
\begin{lstlisting}[language=XML]
<?xml version="1.0" encoding="iso-8859-1"?>
<VALCOMPRAS>
  <VALOR VALUE="500" NUMCOT="1">
    <USERS>
      <USER NAME="EMILY"/>
    </USERS>
  </VALOR>
  <VALOR VALUE="999" NUMCOT="1">
    <USERS>
      <USER NAME="TANIA"/>
    </USERS>
  </VALOR>
  <VALOR VALUE="1000000" NUMCOT="1">
    <USERS>
      <USER NAME="CHRISTIAN"/>
    </USERS>
  </VALOR>
</VALCOMPRAS>
\end{lstlisting}

A est�o algumas consultas XPath aplicadas nesse documento, e se seus resultados:
\begin{description*}
  \item[VALCOMPRAS] Seleciona todos os filhos do elemento raiz (VALCOMPRAS)
  \item[/VALCOMPRAS] Seleciona o elemento raiz
  \item[VALCOMPRAS/VALOR] Seleciona todos os elementos \texttt{VALOR} filhos diretos de \texttt{VALCOMPRAS}
  \item[VALCOMPRAS//USER] Seleciona todos os elementos \texttt{USER} descendentes de \texttt{VALCOMPRAS}. N�o � necess�rio ser um filho direto
  \item[//@VALUE] Seleciona todos os atributos \texttt{VALUE} do documento 
\end{description*}

Para filtrar n�s, faz-se o uso de predicados, express�es entre colchetes. A seguir est�o alguns exemplos de consultas com predicados e seus resultados:
\begin{description*}
	\item[VALCOMPRAS/VALOR[1]]\\
	  Seleciona o primeiro elemento \texttt{VALOR} filho de \texttt{VALCOMPRAS}
	\item[VALCOMPRAS/VALOR[last()]]\\
	  Seleciona o �ltimo elemento \texttt{VALOR} filho de \texttt{VALCOMPRAS}
	\item[VALCOMPRAS/VALOR[position() <= 2]]\\
	  Seleciona os dois primeiros elementos \texttt{VALOR} filhos de \texttt{VALCOMPRAS}
  \item[//VALOR[@VALUE]]\\
    Seleciona todos os elementos \texttt{VALOR} que possuam um atributo \texttt{VALUE}
  \item[//USERS[USER]]\\
    Seleciona todos os elementos \texttt{USERS} que tenham uma tag \texttt{USER} como filho
  \item[VALCOMPRAS/VALOR[@VALUE >= 1000]//USER]]\\
    Seleciona todos os elementos \texttt{USER} que descendentes de um elemento \texttt{VALOR} com o atributo VALUE maior ou igual a 1000 
\end{description*}

\section{XPath Injection}
Al�m de transferir dados em formato XML, muitos desenvolvedores de web services optam por armazenar alguns dados internamente em documentos XML. Utilizando este formato, ao receber uma requisi��o, � realizada uma consulta Xpath sobre esses documentos utilizando os par�metros passados, retornando obtendo a resposta desejada.

O c�digo a seguir, escrito em C\# mostra esse comportamento em um web service que consulta um documento XML parecido com o do exemplo anterior:
\begin{lstlisting}[language=C++]
XmlDocument doc = new XmlDocument();
doc.Load(ConfigurationManager.AppSettings["aprovval.xml"]);
var n= doc.SelectNodes("//VALOR[@VALUE >=" + valor + "]/USERS/USER/@NAME");

var sb = new StringBuilder();

for (int i = 0; i < n.Count; i++)
{
  string s = n.Item(i).Value + ';';
  if(!sb.ToString().Contains(s))
  sb.Append(s);
}
ViewBag.Usuarios = sb.ToString();
return View();
\end{lstlisting}

Na linha 3, � realizada uma consulta XPath baseada em um par�metro para o Web Service. A ideia dessa consulta � obter todos os nomes de usu�rio que podem aprovar cota��es de determinado valor. Supondo que a cota��o de compras seja de R\$1000, essa consulta ficaria como:
\begin{lstlisting}[numbers=none]
//VALOR[@VALUE >= 1000]/USERS/USER/@NAME
\end{lstlisting}
o que retorna a lista de nomes corretamente.

Se este par�metro \texttt{valor} tiver o formato de uma consulta XPath, � poss�vel retornar usu�rios que n�o tenham permiss�o para aprovar este valor. Por exemplo, se o par�metro tiver o formato \texttt{1000 | USERS}, a consulta ficaria como:
\begin{lstlisting}[numbers=none]
 //VALOR[@VALUE >= 1000 | USERS]/USERS/USER/@NAME
\end{lstlisting}
retornando os nomes de todos os usu�rios no documento, independente do valor permitido. Da mesma forma, se a consulta realizada tiver o formado
\begin{lstlisting}[numbers=none]
 //VALOR[@VALUE >= 1000 | //@NAME="EMILY"]/USERS/USER/@NAME
\end{lstlisting}
o nome do usu�rio EMILY ser� retornado independente do valor da cota��o.

A esse tipo de manipula��o da consulta realizada no documento XML � dado o nome \emph{XPath Injection}.
\chapter{Estudo realizado}
Para realiza��o dos experimentos, foi desenvolvida uma aplica��o web contendo diversas vulnerabilidades do tipo \estrangeiro{XPath Injection}. Tamb�m foram utilizadas aplica��es listadas no \emph{Vulnerable Sites Database}, al�m de vulnerabilidades encontradas em estudos de outras pessoas. No total, foram avaliadas trinta e duas vulnerabilidades, sendo dez em \estrangeiro{web services} e as demais aplica��es tradicionais, executadas em um navegador.

\section{Scanners utilizados}
Nesse estudo foram utilizados tr�s scanners de vulnerabilidade: Arachni, w3af e webCruiser. As se��es a seguir descrevem brevemente esses scanners.

\subsection{Arachni}
\estrangeiro{Arachni is a feature-full, modular, high-performance Ruby framework aimed towards helping penetration testers and administrators evaluate the security of web applications. Unlike other scanners, Arachni takes into account the dynamic nature of web applications and can detect changes caused while travelling through the paths of a web application's cyclomatic complexity}\cite{site:arachni}.

Arachni � um framework para auditoria de aplica��es web capaz de detectar grande parte das vulnerabilidades conhecidas. Ele suporta extens�es, ou seja, m�dulos para detec��o de novas vulnerabilidades podem ser desenvolvidos e acoplados ao scanner principal.

Tr�s interfaces principais est�o dispon�veis: linha de comando, interface web e XMLRPC, permitindo que outras aplica��es interajam com o scanner.

Al�m disso, � um projeto de c�digo aberto e bem documentado, o que facilita seu uso e integra��o.

\subsection{w3af}
\estrangeiro{w3af is a Web Application Attack and Audit Framework. The project's goal is to create a framework to find and exploit web application vulnerabilities that is easy to use and extend\cite{site:w3af}.}

O w3af � um scanner de vulnerabilidade e ferramenta de explora��o de servi�os web. Ele � dividido em duas partes: o n�cleo, respons�vel por coordenar o processo de explora��o da aplica��o, e os \estrangeiro{plugins}, que procuram as vulnerabilidades. Mais de 100 plugins est�o dispon�veis\cite{site:w3af:plugins}, capazes de encontrar a maioria dos tipos de vulnerabilidades conhecidos. Dentre os scanners avaliados, foi o �nico capaz de analisar \estrangeiro{web services}.

� um projeto de c�digo aberto e desde 2010 est� sendo desenvolvido em parceria com a Rapid7\cite{w3af:rapid7}, um dos maiores desenvolvedores de solu��es para detec��o de vulnerabilidades em aplica��es web. Com a parceria, a Rapid7 disponibiliza um time para trabalhar no w3af em troca de utilizar parte da tecnologia em seus pr�prios softwares.

\subsection{WebCruiser - Web Vulnerability Scanner}
\estrangeiro{WebCruiser - Web Vulnerability Scanner, an effective and powerful web penetration testing tool that will aid you in auditing your website! It has a Vulnerability Scanner and a series of security tools. It can support scanning website as well as POC (Proof of concept) for web vulnerabilities: SQL Injection, Cross Site Scripting, XPath Injection etc. So, WebCruiser is also an automatic SQL injection tool, an XPath injection tool, and a Cross Site Scripting tool\cite{site:webcruiser}!}

O \emph{WebCruiser} � o �nico scanner comercial dentre os avaliados.


\section{Funcionamento dos scanners}
A detec��o de \estrangeiro{XPath Injection} no Arachni e no w3af funcionam de forma parecida: ele tentam for�ar uma express�o \estrangeiro{XPath} inv�lida e procuram por mensagens de erro que essas express�es possam disparar.

O w3af tenta enviar a string \texttt{d'z``0}, enquanto o Arachni utiliza \texttt{'``}. Nos dois casos, normalmente uma express�o inv�lida ser� gerada. A figura \ref{fig:XPathScanners} mostra uma express�o XPath formada com o ataque do w3af (linha 1) e do Arachni (linha 2) em uma aplica��o desenvolvida para realizar os testes.

\figura{XPathScanners}{Exemplo de express�o XPath gerada com ataque dos scanners Arachni e w3af}{12cm}

No ataque com o Arachni, a express�o fica inv�lida pois uma string � iniciada com o \texttt{``}, mas ela nunca � encerrada (n�o existe um \texttt{''}.

O w3af tamb�m deixa uma string em aberto, come�ando no \texttt{``}, al�m de deixar um caractere ``solto'' (\texttt{z}) na express�o.

O WebCruiser utilizou um \estrangeiro{approach} diferente para detec��o, injetando um conte�do v�lido para a express�o e analisando as requisi��es HTTP para verificar se a vulnerabilidade existe. O conte�do injetado foi \texttt{99999999' or '7'='7}, deixando a express�o XPath v�lida e garantindo que pelo menos um n� sempre seja retornado. A figura \ref{fig:XPathWC} mostra como ficou a express�o XPath com o ataque realizado pelo WebCruiser.

\figura{XPathWC}{Exemplo de express�o XPath gerada com ataque do WebCruiser}{12cm}

\section{Resultados}
Todos os scanners detectaram diversos tipos de vulnerabilidades em todas as aplica��es avaliadas, mas poucas das vulnerabilidades \estrangeiro{XPath injection} foram encontradas, como pode ser observado na tabela \ref{tab:vulnerabilidades}.

\begin{table}[h]
 \centering
 \begin{tabular}{lr}
 \toprule
  \textbf{\estrangeiro{Scanner}} & \textbf{Quantidade} \\
  \midrule
  Arachni & 3 \\
  \midrule
  w3af & 3 \\
  \midrule
  WebCruiser & 5 \\
  \bottomrule
 \end{tabular}
 
 \caption{Vulnerabilidades encontradas}
 \label{tab:vulnerabilidades}
\end{table}

O Arachni e o w3af n�o s� detectaram a mesma quantia de vulnerabilidades, mas detectaram as mesmas tr�s vulnerabilidades. Isso j� era esperado, j� que esses dois scanners possuem implementa��es parecidas para detec��o desse tipo de vulnerabilidade.

O WebCruiser detectou uma quantidade de vulnerabilidades maior, das quais apenas uma foi detectada tamb�m pelos outros scanners.
\chapter{Conclus�o} \label{chap:Conclusao}
Este estudo apresentou os resultados de tr�s scanners de vulnerabilidade na detec��o de uma amostra com mais de trinta vulnerabilidades comprovadas do tipo XPath Injection.

Apesar de os \estrangeiro{scanners} terem conseguido n�meros melhores em outros tipos de vulnerabilidades, pode-se perceber que n�o � t�o simples detectar \estrangeiro{XPath Injection} em uma aplica��o, j� que apenas 15\% das vulnerabilidades foram encontradas.

Os resultados mostram que escolher um \estrangeiro{scanner} de vulnerabilidade n�o � f�cil. Diversos fatores devem ser levados em considera��o, como o tempo necess�rio para varrer a aplica��o, n�mero de resultados encontrados, vulnerabilidades suportadas, etc.

Estudos futuros devem utilizar outros tipos de vulnerabilidades al�m do \estrangeiro{XPath Injection}, e procurar outros \estrangeiro{scanners}.

\bibliography{bibliografia}
\end{document}